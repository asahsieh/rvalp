\chapter{Writing RISC-V Programs}

\enote{Introduce the ISA register names and aliases in here?}%
This chapter introduces each of the RV32I instructions by developing programs 
that demonstrate their usefulness.


%%%%%%%%%%%%%%%%%%%%%%%%%%%%%%%%%%%%%%%%%%%%%%%%%%%%%%%%%%%%%%%%%%%%%%%%%%%%
%%%%%%%%%%%%%%%%%%%%%%%%%%%%%%%%%%%%%%%%%%%%%%%%%%%%%%%%%%%%%%%%%%%%%%%%%%%%
\section{Use {\tt ebreak} to Stop \rvddt{} Execution}
\index{Instruction!ebreak}

The \insn{ebreak} instruction exists for the sole purpose of transferring control back 
to a debugging environment.\cite[p.~24]{rvismv1v22:2017}

When \rvddt{} executes an \insn{ebreak} instruction, it will immediately terminate any
executing {\em trace} or {\em go} command currently executing and return to the
command prompt without advancing the \reg{pc} register.  

The machine language encoding shows that \insn{ebreak} has no operands.

\DrawInsnTypeEPicture{ebreak}{00000000000100000000000001110011}

\listingRef{ebreak/ebreak.out} demonstrates that since \rvddt{} does 
not advance the \reg{pc} when it encounters an \insn{ebreak} instruction, 
subsequent {\em trace} and/or {\em go} commands will re-execute the same \insn{ebreak} 
and halt the simulation again (and again).  
This feature is intended to help prevent overzealous users from accidently 
running past the end of a code fragment.\footnote{This was one of the first {\em enhancements}
I needed for myself \tt:-)}

\listing{ebreak/ebreak.S}{A one-line \insn{ebreak} program.}

\listing{ebreak/ebreak.out}{\insn{ebreak} stopps \rvddt{} without advancing \reg{pc}.}


%%%%%%%%%%%%%%%%%%%%%%%%%%%%%%%%%%%%%%%%%%%%%%%%%%%%%%%%%%%%%%%%%%%%%%%%%%%%
%%%%%%%%%%%%%%%%%%%%%%%%%%%%%%%%%%%%%%%%%%%%%%%%%%%%%%%%%%%%%%%%%%%%%%%%%%%%
%%%%%%%%%%%%%%%%%%%%%%%%%%%%%%%%%%%%%%%%%%%%%%%%%%%%%%%%%%%%%%%%%%%%%%%%%%%%
\section{Using the \insn{addi} Instruction}
\index{Instruction!addi}

\enote{Define what constant and immediate values are somewhere.}%
The detailed description of how the \insn{addi} instruction is executed
is that it:
\begin{enumerate}
\item Sign-extends the immediate operand.
\item Add the sign-extended immediate operand to the contents of the \reg{rs1} register.
\item Store the sum in the \reg{rd} register.
\item Add four to the \reg{pc} register (point to the next instruction.)
\end{enumerate}

In the following example \reg{rs1} = \reg{x28}, \reg{rd} = \reg{x29} and
the immediate operand is -1.

\DrawInsnTypeIPicture{addi x29, x28, -1}{11111111111111100000111010010011}

Depending on the values of the fields in this instruction a number of
different operations can be performed.  The most obvious is that it
can add things.  But it can also be used to copy registers, set a 
register to zero and even, when you need to, accomplish nothing.

%%%%%%%%%%%%%%%%%%%%%%%%%%%%%%%%%%%%%%%%%%%%%%%%%%%%%%%%%%%%%%%%%%%%%%%%%%%%
%%%%%%%%%%%%%%%%%%%%%%%%%%%%%%%%%%%%%%%%%%%%%%%%%%%%%%%%%%%%%%%%%%%%%%%%%%%%
\subsection{No Operation}
\index{Instruction!nop}

It might seem odd but it is sometimes important to be able to execute
an instruction that accomplishes nothing while simply advancing the 
\reg{pc} to the next instruction.  One reason for this is to fill 
unused memory between two instructions in a program.%
\footnote{This can happen during the evolution of one portion of code 
that reduces in size but has to continue to fit into a system without 
altering any other code\ldots\ or some times you just need to waste 
a small amount of time in a device driver.}

An instruction that accomplishes nothing is called a \insn{nop}
(some times systems call these \insn{noop}).  The name means 
{\em no operation}.  
The intent of a \insn{nop} is to execute without having any side effects 
other than to advance the \reg{pc} register.

The \insn{addi} instruction can serve as a \insn{nop} by coding it like this:

\DrawInsnTypeIPicture{addi x0, x0, 0}{00000000000000000000000000010011}

The result will be to add zero to zero and discard the result (because you
can never store a value into the x0 register.)

The RISC-V assembler provides a pseudoinstruction specifically for this 
purpose that you can use to improve the readability of your code.  Note 
that the \insn{addi} and \insn{nop} instructions in \listingRef{nop/nop.S}
are assembled into the exact same binary machine instruction (The
\hex{00000013} you can see are stored at addresses \hex{0} and \hex{4})
as seen by looking at the objdump listing in \listingRef{nop/nop.lst}.
In fact, you can see that objdump shows both instructions as a \insn{nop}
while \listingRef{nop/nop.out} shows that \rvddt{} displays both as 
\verb@addi x0, x0, 0@.  

\listing{nop/nop.S}{Demonstrate that an \insn{addi} can be the same as \insn{nop}.}

\index{objdump}
\listing{nop/nop.lst}{Using \insn{addi} to perform a \insn{nop}}

\listing{nop/nop.out}{Using \insn{addi} to perform a \insn{nop}}


%%%%%%%%%%%%%%%%%%%%%%%%%%%%%%%%%%%%%%%%%%%%%%%%%%%%%%%%%%%%%%%%%%%%%%%%%%%%
%%%%%%%%%%%%%%%%%%%%%%%%%%%%%%%%%%%%%%%%%%%%%%%%%%%%%%%%%%%%%%%%%%%%%%%%%%%%
\subsection{Copying the Contents of One Register to Another}

By adding zero to one register and storing the sum in another register
the \insn{addi} instruction can be used to copy the value stored in one
register to another register.  The following instruction will copy
the contents of \reg{t4} into \reg{t3}.

\DrawInsnTypeIPicture{addi t3, t4, 0}{00000000000011101000111000010011}

\index{Instruction!mv}
This is a commonly required operation.  To make your intent clear
you may use the \insn{mv} pseudoinstruction for this purpose.  

\listingRef{mv/mv.S} shows the source of a program that is dumped in 
\listingRef{mv/mv.lst} illustrating that the assembler has generated the
same machine instruction (\hex{000e8e13} at addresses \hex{0} and \hex{4}) 
for both of the instructions.

\listing{mv/mv.S}{Comparing \insn{addi} to \insn{mv}}

\listing{mv/mv.lst}{An objdump of an \insn{addi} and \insn{mv} Instruction.}

%%%%%%%%%%%%%%%%%%%%%%%%%%%%%%%%%%%%%%%%%%%%%%%%%%%%%%%%%%%%%%%%%%%%%%%%%%%%
%%%%%%%%%%%%%%%%%%%%%%%%%%%%%%%%%%%%%%%%%%%%%%%%%%%%%%%%%%%%%%%%%%%%%%%%%%%%
\subsection{Setting a Register to Zero}

Recall that \reg{x0} always contains the value zero.  Any register
can be set to zero by copying the contents of \reg{x0} using \insn{mv} 
(aka \insn{addi}).%
\footnote{There are other pseudoinstructions (such as \insn{li}) that can also 
turn into an \insn{addi} instruction.  Objdump might display `{\tt addi t3,x0,0}'
as `{\tt mv t3,x0}' or `{\tt li t3,0}'.}

For example, to set \reg{t3} to zero: 

\DrawInsnTypeIPicture{addi t3, x0, 0}{00000000000000000000111000010011}

\listing{mvzero/mv.S}{Using \insn{mv} (aka \insn{addi}) to zero-out a register.}

\listingRef{mvzero/mv.out} traces the execution of the program in 
\listingRef{mvzero/mv.S} showing how \reg{t3} is changed from \hex{f0f0f0f0} 
(seen on $\ell 16$) to \hex{00000000} (seen on $\ell 26$.)

\listing{mvzero/mv.out}{Setting \reg{t3} to zero.}



%%%%%%%%%%%%%%%%%%%%%%%%%%%%%%%%%%%%%%%%%%%%%%%%%%%%%%%%%%%%%%%%%%%%%%%%%%%%
%%%%%%%%%%%%%%%%%%%%%%%%%%%%%%%%%%%%%%%%%%%%%%%%%%%%%%%%%%%%%%%%%%%%%%%%%%%%
\subsection{Adding a 12-bit Signed Value}


\DrawInsnTypeIPicture{addi x1, x7, 4}{00000000010000111000000010010011}





\begin{verbatim}
    addi    t0, zero, 4     # t0 = 4
    addi    t1, t1, 100     # t1 = 104

    addi    t0, zero, 0x123     # t0 = 0x123
    addi    t0, t0, 0xfff       # t0 = 0x122 (subtract 1)

    addi    t0, zero, 0xfff     # t0 = 0xffffffff (-1)  (diagram out the chaining carry)
                                # refer back to the overflow/truncation discussion in binary chapter

	addi x0, x0, 0				# no operation (pseudo: nop)
	addi rd, rs, 0				# copy reg rs to rd (pseudo: mv rd, rs)
\end{verbatim}

Demonstrate various addi instructions. 


%%%%%%%%%%%%%%%%%%%%%%%%%%%%%%%%%%%%%%%%%%%%%%%%%%%%%%%%%%%%%%%%%%%%%%%%%%%%
%%%%%%%%%%%%%%%%%%%%%%%%%%%%%%%%%%%%%%%%%%%%%%%%%%%%%%%%%%%%%%%%%%%%%%%%%%%%
\section{todo}

Ideas for the order of introducing instructions.


\section{Other Instructions With Immediate Operands}
\begin{verbatim}
    addi
    andi
    ori
    xori

    slti
    sltiu
    srai
    slli
    srli
\end{verbatim}


\section{Transferring Data Between Registers and Memory}

RV is a load-store architecture.  This means that the only way that the
CPU can interact with the memory is via the {\em load} and {\em store}
instructions.  All other data manipulation must be performed on register
values.

Copying values from memory to a register (first examples using regs set with addi):
\begin{verbatim}
    lb
    lh
    lw
    lbu
    lhu
\end{verbatim}

Copying values from a register to memory:
\begin{verbatim}
    sb
    sh
    sw
\end{verbatim}

\enote{Mention the rvddt UART I/O address for writing to the console here?}

\section{RR operations}
\begin{verbatim}
    add
    sub
    and
    or
    sra
    srl
    sll
    xor
    sltu
    slt
\end{verbatim}


\section{Setting registers to large values using lui with addi}

\begin{verbatim}
    addi        // useful for values from -2048 to 2047
    lui         // useful for loading any multiple of 0x1000

    Setting a register to any other value must be done using a combo of insns:

    auipc       // Load an address relative the the current PC (see la pseudo)
    addi


    lui         // Load constant into into bits 31:12  (see li pseudo)
    addi        // add a constant to fill in bits 11:0
                    if bit 11 is set then need to +1 the lui value to compensate
\end{verbatim}


\section{Labels and Branching}

Start to introduce addressing here?

\begin{verbatim}
    beq
    bne
    blt
    bge
    bltu
    bgeu

    bgt rs, rt, offset      # pseudo for: blt rt, rs, offset    (reverse the operands)
    ble rs, rt, offset      # pseudo for: bge rt, rs, offset    (reverse the operands)
    bgtu rs, rt, offset     # pseudo for: bltu rt, rs, offset   (reverse the operands)
    bleu rs, rt, offset     # pseudo for: bgeu rt, rs, offset   (reverse the operands)

    beqz beqz rs, offset    # pseudo for: beq rs, x0, offset
    bnez rs, offset         # pseudo for: bne rs, x0, offset
    blez rs, offset         # pseudo for: bge x0, rs, offset
    bgez rs, offset         # pseudo for: bge rs, x0, offset
    bltz rs, offset         # pseudo for: blt rs, x0, offset
    bgtz rs, offset         # pseudo for: blt x0, rs, offset
\end{verbatim}


\section{Relocation}

\begin{verbatim}
Absolute:
    %hi(symbol)         
    %lo(symbol)         

PC-relative:
    %pcrel_hi(symbol)   
    %pcrel_lo(label)    

Using the auipc & addi pair with label references:
    The %pcrel_lo() uses the label to find the associated %pcrel_hi()
    The label MUST be on a line that used a %pcrel_hi() or get an error.
    This is needed to calculate the proper offset.
    Things like this are legal (though not sure of the value):
    label:  auipc   t1, %pcrel_hi(symbol)
        addi    t2, t1, %pcrel_lo(label)
        addi    t3, t1, %pcrel_lo(label)
        lw      t4, %pcrel_lo(label)(t1)
        sw      t5, %pcrel_lo(label)(t1)

Discuss how relaxation works.
see:  https://github.com/riscv/riscv-elf-psabi-doc/blob/master/riscv-elf.md
\end{verbatim}


\section{Jumps}

Introduce and present subroutines but not nesting until introduce stack operations.

\begin{verbatim}
    jal
    jalr
\end{verbatim}



\section{Pseudo Operations}

\enote{Explain why we have pseudo ops.   These mappings are lifted from the ISM, Vol 1, V2.2}%
\begin{verbatim}
    la rd, symbol
                    auipc rd, symbol[31:12]
                    addi rd, rd, symbol[11:0]

    l{b|h|w|d} rd, symbol
                    auipc rd, symbol[31:12]
                    l{b|h|w|d} rd, symbol[11:0](rd)

    s{b|h|w|d} rd, symbol, rt                   # rt is the temp reg to use for the operation
                    auipc rt, symbol[31:12]
                    s{b|h|w|d} rd, symbol[11:0](rt)


    j offset        jal x0, offset
    jal offset      jal x1, offset
    jr rs           jalr x0, rs, 0
    jalr rs         jalr x1, rs, 0
    ret             jalr x0, x1, 0

    call offset     auipc x6, offset[31:12]
                    jalr x1, x6, offset[11:0]

    tail offset     auipc x6, offset[31:12]     # same as call but no x1
                    jalr x0, x6, offset[11:0]
\end{verbatim}


\section{The Linker and Relaxation}

\enote{Needs research.
I'm not sure if/how the linker alone can relax the AUIPC+JALR pair since 
the assembler could have used a pcrel branch across one of these pairs.}%
I don't know where this should go just yet.

\section{pic and nopic}

pic is {\em needed} for shared libs.  Should discuss it but probably best 
to leave the topic for a later chapter.
