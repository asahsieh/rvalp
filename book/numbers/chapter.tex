\chapter{Number Systems}
\label{chapter:NumberSystems}

RISC-V systems represent information using binary values stored in 
little-endian order.\footnote{See\cite{IEN137} for some history of 
the big/little-endian ``controversy.''}

\section{Integers}

A binary integer is constructed with only 1s and 0s in the same
manner as decimal numbers are constructed with values from 0 to 9.

Counting in binary is the same as in decimal.  For example, when
adding 1 to 9, the carry is added to the next place value.  When
subtracting 1 from 0, a borrow is required and so on.

Figure~\autoref{Figure:integers} shows an abridged table of the 
decimal, binary and hexadecimal values from 0 to 129.

\begin{figure}[ht]
\begin{center}
\begin{tabular}{|c|c|c||c|c|c|c|c|c|c|c||c|c|}
\hline
\multicolumn{3}{|c||}{Decimal} & \multicolumn{8}{|c||}{Binary} & \multicolumn{2}{|c|}{Hex}\\
\hline
$10^2$	& $10^1$	& $10^0$	& $2^7$ & $2^6$ & $2^5$ & $2^4$ & $2^3$ & $2^2$ & $2^1$ & $2^0$		& $16^1$ & $16^0$	\\
\hline
100		&	10		& 1			& 128	& 64	& 32	& 16	& 8		& 4		& 2		& 1			& 16	& 1			\\
\hline \hline
0		&	0		&	0		& 0 & 0 & 0 & 0 & 0 & 0 & 0 & 0		& 0 & 0 \\
0		&	0		&	1		& 0 & 0 & 0 & 0 & 0 & 0 & 0 & 1		& 0 & 1 \\
0		&	0		&	2		& 0 & 0 & 0 & 0 & 0 & 0 & 1 & 0		& 0 & 2 \\
0		&	0		&	3		& 0 & 0 & 0 & 0 & 0 & 0 & 1 & 1		& 0 & 3 \\
0		&	0		&	4		& 0 & 0 & 0 & 0 & 0 & 1 & 0 & 0		& 0 & 4 \\
0		&	0		&	5		& 0 & 0 & 0 & 0 & 0 & 1 & 0 & 1		& 0 & 5 \\
0		&	0		&	6		& 0 & 0 & 0 & 0 & 0 & 1 & 1 & 0		& 0 & 6 \\
0		&	0		&	7		& 0 & 0 & 0 & 0 & 0 & 1 & 1 & 1		& 0 & 7 \\
0		&	0		&	8		& 0 & 0 & 0 & 0 & 1 & 0 & 0 & 0		& 0 & 8 \\
0		&	0		&	9		& 0 & 0 & 0 & 0 & 1 & 0 & 0 & 1		& 0 & 9 \\
0		&	1		&	0		& 0 & 0 & 0 & 0 & 1 & 0 & 1 & 0		& 0 & a \\
0		&	1		&	1		& 0 & 0 & 0 & 0 & 1 & 0 & 1 & 1		& 0 & b \\
0		&	1		&	2		& 0 & 0 & 0 & 0 & 1 & 1 & 0 & 0		& 0 & c \\
0		&	1		&	3		& 0 & 0 & 0 & 0 & 1 & 1 & 0 & 1		& 0 & d \\
0		&	1		&	4		& 0 & 0 & 0 & 0 & 1 & 1 & 1 & 0		& 0 & e \\
0		&	1		&	5		& 0 & 0 & 0 & 0 & 1 & 1 & 1 & 1		& 0 & f \\
0		&	1		&	6		& 0 & 0 & 0 & 1 & 0 & 0 & 0 & 0		& 1 & 0 \\
0		&	1		&	7		& 0 & 0 & 0 & 1 & 0 & 0 & 0 & 1		& 1 & 1 \\
\hline
\multicolumn{3}{|c||}{\ldots} & \multicolumn{8}{|c||}{\ldots} & \multicolumn{2}{|c|}{\ldots}\\
\hline
1       &   2       &   5		& 0 & 1 & 1 & 1 & 1 & 1 & 0 & 1		& 7 & d \\
1       &   2       &   6		& 0 & 1 & 1 & 1 & 1 & 1 & 1 & 0		& 7 & e \\
1       &   2       &   7		& 0 & 1 & 1 & 1 & 1 & 1 & 1 & 1		& 7 & f \\
1       &   2       &   8		& 1 & 0 & 0 & 0 & 0 & 0 & 0 & 0		& 8 & 0 \\
\hline
\end{tabular}
\end{center}
\captionof{figure}{Counting in decimal, binary and hexadecimal.}
\label{Figure:integers}
\end{figure}


One way to look at this table is on a per--row basis where each place 
value is represented by the base raised to the power of the place value
position (shown in the column headings.)  This is useful when
converting arbitrary values between bases.  For example to interpret
the decimal value on the fourth row: 
\[ 0 \times 10^2 + 0 \times 10^1 + 3 \times 10^0 = 3_{10} \]
And to interpret binary value on the same row by converting it to decimal:
\[ 0 \times 2^7 + 0 \times 2^6 +0 \times 2^5 +0 \times 2^4 +0 \times 2^3 +0 \times 2^2 + 1 \times 2^1 + 1 \times 2^0 = 3_{10} \]
And the same for the hexadecimal value:
\[ 0 \times 16^1 + 3 \times 16^0 = 3_{10} \]


Another way to look at this table is on a per--column basis.  When
tasked with drawing such a table by hand, it might be useful 
to observe that, just as in decimal, the right--most column will
cycle through all of the values represented in the chosen base
then cycle back to zero and repeat.  (For example, in binary this
pattern is 0-1-0-1-0-1-0-\ldots) The next column in each base
will cycle in the same manner except each of the values is repeated
as many times as is represented by the place value (in the case of 
decimal, $10^1$ times, binary $2^1$ times, hex $16^1$ times.  Again,
the for binary numbers this pattern is  0-0-1-1-0-0-1-1-\ldots)
This continues for as many columns as are needed to represent the 
magnitude of the desired number.

Another item worth noting is that any even binary number will always
have a 0 LSB and odd numbers will always have a 1 LSB.

As is customary in decimal, leading zeros are sometimes not shown 
for readability.

The relationship between binary and hex values is also worth taking
note.  Because $2^4 = 16$, there is a clean and simple grouping
of 4 \gls{bit}s to 1 \gls{hit}.  There is no such relationship
between binary and decimal.  

Writing and reading numbers in binary that are longer than 8 bits 
is cumbersome and prone to error.  The simple conversion between 
binary and hex makes hex a convenient shorthand for expressing 
binary values in many situations.  

For example, consider the following value expressed in binary, 
hexadecimal and decimal (spaced to show the relationship
between binary and hex):

\begin{verbatim}
Binary value:       0010 0111 1011 1010 1100 1100 1111 0101
Hex Value:             2    7    B    A    C    C    F    5
Decimal Value:                                    666553589
\end{verbatim}

Empirically we can see that grouping the bits into sets of four
allows an easy conversion to hex and expressing it as such is
$\frac{1}{4}$ as long as in binary while at the same time
allowing for easy conversion back to binary.

The decimal value in this example does not easily convey a sense
of the binary value.

%%%%%%%%%%%%%%%%%%%%%%%%%%%%%%%%%%%%%%%%%%%%%%%%%%%%%%%%%%%%%%%%%%%%%%%%%
%%%%%%%%%%%%%%%%%%%%%%%%%%%%%%%%%%%%%%%%%%%%%%%%%%%%%%%%%%%%%%%%%%%%%%%%%%
\subsection{Converting Between Bases}

%%%%%%%%%%%%%%%%%%%%%%%%%%%%%%%%%%%%%%%%%%%%%%%%%%%%%%%%%%%%%%%%%%%%%%%%%%
\subsubsection{From Binary to Decimal}
\label{section:bindec}

Alas, it is occasionally necessary to convert between decimal,
binary and/or hex.

To convert from binary to decimal, put the decimal value of the place values 
{\ldots8 4 2 1} over the binary digits like this:

\begin{verbatim}
    128 64 32 16  8  4  2  1
      0  0  0  1  1  0  1  1
\end{verbatim}

Now sum the place--values that are expressed in decimal for each 
bit with the value of 1: $16+8+2+1$.  The integer binary value
$00011011_2$ represents the decimal value $27_{10}$.

%%%%%%%%%%%%%%%%%%%%%%%%%%%%%%%%%%%%%%%%%%%%%%%%%%%%%%%%%%%%%%%%%%%%%%%%%%
\subsubsection{From Binary to Hexadecimal}
\label{section:binhex}

Conversion from binary to hex involves grouping the bits into
sets of four and then performing the same summing process as 
shown above.  If there is not a multiple of four bits then
extend the binary to the left with zeros to make it so.

Grouping the bits into sets of four and summing:

\begin{verbatim}
Place:         8 4 2 1     8 4 2 1     8 4 2 1     8 4 2 1
Binary:        0 1 1 0     1 1 0 1     1 0 1 0     1 1 1 0
Decimal:         4+2  =6   8+4+  1=13  8+  2  =10  8+4+2  =14
\end{verbatim}

After the summing, convert each decimal value to hex.  The decimal
values from 0--9 are the same values in hex.  Because we don't have any
more numerals to represent the values from 10-15, we use the first 6
letters (See the right--most column of \autoref{Figure:integers}.)
Fortunately there are only six hex mappings involving letters.  Thus 
it is reasonable to memorize them.

Continuing this example:

\begin{verbatim}
Decimal:               6           13          10          14
Hex:                   6           D           A           E
\end{verbatim}



%%%%%%%%%%%%%%%%%%%%%%%%%%%%%%%%%%%%%%%%%%%%%%%%%%%%%%%%%%%%%%%%%%%%%%%%%%
\subsubsection{From Hexadecimal to Binary}

Again, the four--bit mapping between binary and hex makes this
task as straight forward as using a look-up table.

For each \gls{hit} (Hex digIT), translate it to its unique four--bit pattern.
Perform this task either by memorizing each of the 16 patterns 
or by converting each hit to decimal first and then converting
each four--bit binary value to decimal using the place--value summing 
method discussed in \autoref{section:bindec}.

For example:

\begin{verbatim}
Hex:           4            C
Binary:        0  1  0  0   1 1 0 0
Decimal:     128 64 32 16   8 4 2 1
Sum:             64+        8+4       = 76
\end{verbatim}


%%%%%%%%%%%%%%%%%%%%%%%%%%%%%%%%%%%%%%%%%%%%%%%%%%%%%%%%%%%%%%%%%%%%%%%%%%
\subsubsection{From Decimal to Binary}

To convert arbitrary decimal numbers to binary, extend the list 
of binary place values until it exceeds the value of the decimal 
number being converted.  Then make successive subtractions of each 
of the place values that would yield a non-negative result.

For example, to convert $1234_{10}$ to binary:

\begin{verbatim}
Place values: 2048-1024-512-256-128-64-32-16-8-4-2-1

   0          2048      (too big)
   1   1234 - 1024 = 210
   0          512       (too big)
   0          256       (too big)
   1    210 - 128  = 82
   1     82 - 64   = 18
   0          32        (too big)
   1     18 - 16   = 2
   0          8         (too big)
   0          4         (too big)
   1      2 - 2    = 0
   0          1         (too big)
\end{verbatim}

The answer using this notation is listed vertically
in the left column with the \acrshort{msb} on the top and 
the \acrshort{lsb} on the bottom line: $010011010010_2$.


%%%%%%%%%%%%%%%%%%%%%%%%%%%%%%%%%%%%%%%%%%%%%%%%%%%%%%%%%%%%%%%%%%%%%%%%%%
\subsubsection{From Decimal to Hex}

Conversion from decimal to hex can be done by using the place
values for base--16 and the same math as from decimal to binary
or by first converting the decimal value to binary and then
from binary to hex by using the methods discussed above.

Because binary and hex are so closely related, performing
a conversion by way of binary is quite straight forward.


%%%%%%%%%%%%%%%%%%%%%%%%%%%%%%%%%%%%%%%%%%%%%%%%%%%%%%%%%%%%%%%%%%%%%%%%%
%%%%%%%%%%%%%%%%%%%%%%%%%%%%%%%%%%%%%%%%%%%%%%%%%%%%%%%%%%%%%%%%%%%%%%%%%%
\subsection{Addition of Binary Numbers}

The addition of binary numbers can be performed long--hand the
same way decimal addition is taught in grade school.  In fact binary
addition is easier since it only involves adding 0 or 1.

The first thing to note that in any number base $0+0=0$, $0+1=1$, and 
$1+0=1$.  Since there is no ``two'' in binary (just like there is 
no ``ten'' decimal) adding $1+1$ results in a zero with a carry as
in: $1+1=10_2$ and in: $1+1+1=11_2$.  Using these five sums, any two
binary integers can be added.

For example:

\begin{verbatim}
        111111  1111  <== carries
     0110101111001111 <== addend
   + 0000011101100011 <== addend
   ------------------
     0111001100110010 <== sum
\end{verbatim}

%%%%%%%%%%%%%%%%%%%%%%%%%%%%%%%%%%%%%%%%%%%%%%%%%%%%%%%%%%%%%%%%%%%%%%%%%%
%%%%%%%%%%%%%%%%%%%%%%%%%%%%%%%%%%%%%%%%%%%%%%%%%%%%%%%%%%%%%%%%%%%%%%%%%
\subsection{Signed Numbers}

There are multiple methods used to represent signed binary integers.
The method used by most modern computers is called ``two's complement.''

A two's complement number is encoded in such a manner as to simplify
the hardware used to add, subtract and compare integers.  

A simple method of thinking about two's complement numbers is to
negate the place value of the \acrshort{msb}.  For example, the 
number one is represented the same as discussed before:

\begin{verbatim}
   -128 64 32 16  8  4  2  1
      0  0  0  0  0  0  0  1
\end{verbatim}

The \acrshort{msb} of any negative number in this format will always
be 1.  For example the value $-1_{10}$ is:

\begin{verbatim}
   -128 64 32 16  8  4  2  1
      1  1  1  1  1  1  1  1
\end{verbatim}

\ldots because: $-128+64+32+16+8+4+2+1=-1$.

This format has the virtue of allowing the same addition logic 
discussed above to be used to calculate $-1+1=0$.

\begin{verbatim}
   -128 64 32 16  8  4  2  1 <== place value
   1  1  1  1  1  1  1  1  0 <== carries
      1  1  1  1  1  1  1  1 <== addend (-1)
    + 0  0  0  0  0  0  0  1 <== addend (1)
      ----------------------
   1  0  0  0  0  0  0  0  0 <== sum (0 with an overflow)
\end{verbatim}

In order for this to work, the \gls{overflow} carry out of the 
sum of the MSBs is ignored.

%%%%%%%%%%%%%%%%%%%%%%%%%%%%%%%%%%%%%%%%%%%%%%%%%%%%%%%%%%%%%%%%%%%%%%%%%%
\subsubsection{Converting between Positive and Negative}

Changing the sign on two's complement numbers can be described as
inverting all of the bits (which is also known as the one's complement)
and then add one.

For example, inverting the number {\em four}:

\begin{verbatim}
   -128 64 32 16  8  4  2  1
      0  0  0  0  0  1  0  0 <== 4

                     1  1    <== carries
      1  1  1  1  1  0  1  1 <== one's complement of 4
    + 0  0  0  0  0  0  0  1 <== plus 1
      ----------------------
      1  1  1  1  1  1  0  0 <== -4
\end{verbatim}

This can be verified by adding 5 to the result and observe that
the sum is 1:

\begin{verbatim}
   -128 64 32 16  8  4  2  1
      1  1  1  1  1          <== carries
      1  1  1  1  1  1  0  0 <== -4
    + 0  0  0  0  0  1  0  1 <== 5
      ----------------------
   1  0  0  0  0  0  0  0  1
\end{verbatim}

Note that the changing of the sign using this method is symmetric
in that it is identical when converting from negative to positive
and when converting from positive to negative: flip the bits and
add 1.

For example, changing the value -4 to 4 to illustrate the
reverse of the conversion above:

\begin{verbatim}
   -128 64 32 16  8  4  2  1
      1  1  1  1  1  1  0  0 <== -4

                     1  1    <== carries
      0  0  0  0  0  0  1  1 <== one's complement of -4
    + 0  0  0  0  0  0  0  1 <== plus 1
      ----------------------
      0  0  0  0  0  1  0  0 <== 4
\end{verbatim}

%%%%%%%%%%%%%%%%%%%%%%%%%%%%%%%%%%%%%%%%%%%%%%%%%%%%%%%%%%%%%%%%%%%%%%%%%
%%%%%%%%%%%%%%%%%%%%%%%%%%%%%%%%%%%%%%%%%%%%%%%%%%%%%%%%%%%%%%%%%%%%%%%%%%
\subsection{Subtraction of Binary Numbers}


Subtraction of binary numbers is performed by first negating
the subtrahend and then adding the two numbers.  Due to the
nature of two's complement numbers this will work for both 
signed and unsigned numbers.
\enote{This section needs more examples of subtracting 
signed an unsigned numbers and a discussion on how 
signedness is not relevant until the results are interpreted. 
For example adding $-4+ -8=-12$ using two 8--bit numbers 
is the same as adding $252+248=500$ and truncating the result 
to 244.}

To calculate $-4-8 = -12$

\begin{verbatim}
   -128 64 32 16  8  4  2  1
      1  1  1  1  1  1  0  0 <== -4
    - 0  0  0  0  1  0  0  0 <== 8


                  1  1  1    <== carries
      1  1  1  1  0  1  1  1 <== one's complement of -8
    + 0  0  0  0  0  0  0  1 <== plus 1
      ----------------------
      1  1  1  1  1  0  0  0 <== -8
      
	  
      1  1  1  1             <== carries
      1  1  1  1  1  1  0  0 <== -4
    + 1  1  1  1  1  0  0  0 <== -8
      ----------------------
   1  1  1  1  1  0  1  0  0 < == -12
\end{verbatim}


%%%%%%%%%%%%%%%%%%%%%%%%%%%%%%%%%%%%%%%%%%%%%%%%%%%%%%%%%%%%%%%%%%%%%%%%%
%%%%%%%%%%%%%%%%%%%%%%%%%%%%%%%%%%%%%%%%%%%%%%%%%%%%%%%%%%%%%%%%%%%%%%%%%
\subsection{Truncation and Overflow}

Disscuss the details of truncation and overflow here.
\enote{This chapter should be made consistent in its use of 
{\em truncation} and {\em overflow} as occur with signed and unsigned
addition and subtraction.}

%%%%%%%%%%%%%%%%%%%%%%%%%%%%%%%%%%%%%%%%%%%%%%%%%%%%%%%%%%%%%%%%%%%%%%%%%
%%%%%%%%%%%%%%%%%%%%%%%%%%%%%%%%%%%%%%%%%%%%%%%%%%%%%%%%%%%%%%%%%%%%%%%%%
\subsection{Logical/Boolean Functions}

Unlike addition and subtraction, boolean functions apply 
on a per--bit basis.
%in that they do not impact neighboring bits.
%by generating things like a carry or a borrow.
When applied to multi--bit values, each bit position is operated upon 
independantly of the other bits.
\enote{This is unclear.  Need to define bit positions and probably
should add basic truth table diagrams.}
\enote{Need to define 1 as true and 0 as false somewhere.}

%%%%%%%%%%%%%%%%%%%%%%%%%%%%%%%%%%%%%%%%%%%%%%%%%%%%%%%%%%%%%%%%%%%%%%%%%
%%%%%%%%%%%%%%%%%%%%%%%%%%%%%%%%%%%%%%%%%%%%%%%%%%%%%%%%%%%%%%%%%%%%%%%%%%
\subsubsection{NOT}

The {\em NOT} operator applies to a single operand and represents the 
opposite of the input.
\enote{Need to define unary, binary and ternary operators without 
confusing binary operators with binary numbers.}

If the input is 1 then the output is 0.  If the input is 0 then the
output is 1.  In other words, the output value is {\em not} that of the 
input value.

This text will use the operator used in the C language when discussing 
the {\em NOT} operator in symbolic form.  Specifically the tilde: `\verb@~@'.

\begin{verbatim}
  ~ 1 1 1 1 0 1 0 1  <== A
  -----------------
    0 0 0 0 1 0 1 0  <== output
\end{verbatim}

In a line of code the above might read like this: \verb@output = ~A@

%%%%%%%%%%%%%%%%%%%%%%%%%%%%%%%%%%%%%%%%%%%%%%%%%%%%%%%%%%%%%%%%%%%%%%%%%
%%%%%%%%%%%%%%%%%%%%%%%%%%%%%%%%%%%%%%%%%%%%%%%%%%%%%%%%%%%%%%%%%%%%%%%%%%
\subsubsection{AND}

The boolean {\em and} function has two or more inputs and the output is a
single bit.  The output is 1 if and only if all of the input values are 1.
Otherwise it is 0.

This text will use the operator used in the C language when discussing 
the {\em AND} operator in symbolic form.  Specifically the ampersand: `\verb@&@'.

This function works like it does in spoken language.  For example
if A is 1 {\em AND} B is 1 then the output is 1 (true).
Otherwise the output is 0 (false).  For example:

\begin{verbatim}
    1 1 1 1 0 1 0 1  <== A
  & 1 0 0 1 0 0 1 1  <== B
  -----------------
    1 0 0 1 0 0 0 1  <== output
\end{verbatim}

In a line of code the above might read like this: \verb@output = A & B@

%%%%%%%%%%%%%%%%%%%%%%%%%%%%%%%%%%%%%%%%%%%%%%%%%%%%%%%%%%%%%%%%%%%%%%%%%
%%%%%%%%%%%%%%%%%%%%%%%%%%%%%%%%%%%%%%%%%%%%%%%%%%%%%%%%%%%%%%%%%%%%%%%%%%
\subsubsection{OR}

The boolean {\em or} function has two or more inputs and the output is a
single bit.  The output is 1 if at least one of the input values are 1.

This text will use the operator used in the C language when discussing
the {\em OR}  operator in symbolic form.  Specifically the pipe: `\verb@|@'.

This function works like it does in spoken language.  For example
if A is 1 {\em OR} B is 1 then the output is 1 (true).
Otherwise the output is 0 (false).  For example:

\begin{verbatim}
    1 1 1 1 0 1 0 1  <== A
  | 1 0 0 1 0 0 1 1  <== B
  -----------------
    1 1 1 1 0 1 1 1  <== output
\end{verbatim}

In a line of code the above might read like this: \verb@output = A | B@

%%%%%%%%%%%%%%%%%%%%%%%%%%%%%%%%%%%%%%%%%%%%%%%%%%%%%%%%%%%%%%%%%%%%%%%%%
%%%%%%%%%%%%%%%%%%%%%%%%%%%%%%%%%%%%%%%%%%%%%%%%%%%%%%%%%%%%%%%%%%%%%%%%%%
\subsubsection{XOR}

The boolean {\em exclusive or} function has two or more inputs and the 
output is a single bit.  The output is 1 if only an odd number of inputs 
are 1. Otherwise the output will be 0.

This text will use the operator used in the C language when discussing
the {\em XOR}  operator in symbolic form.  Specifically the carrot: `\verb@^@'.

Note that when {\em XOR} is used with two inputs, the output
is set to 1 (true) when the inputs have different values and 0 
(false) when the inputs both have the same value.

For example:

\begin{verbatim}
    1 1 1 1 0 1 0 1  <== A
  ^ 1 0 0 1 0 0 1 1  <== B
  -----------------
    0 1 1 0 0 1 1 0  <== output
\end{verbatim}

In a line of code the above might read like this: \verb@output = A ^ B@



%\section{Context}
%
%Numbers can be interpreted differently depending on the context in 
%which they are used.  For example a number may represent the quantity 
%of millimeters between two points.  It may enumerate  a 
%a letter of the alphabet -- ie.  $01000001=A$, $01000010=B$, 
%$01000011=C$\ldots\ In fact, any finite set of items can be identified 
%(enumerated) by a assigning a code number to each element in this fashon.



%%%%%%%%%%%%%%%%%%%%%%%%%%%%%%%%%%%%%%%%%%%%%%%%%%%%%%%%%%%%%%%%%%%%%%%%%
%%%%%%%%%%%%%%%%%%%%%%%%%%%%%%%%%%%%%%%%%%%%%%%%%%%%%%%%%%%%%%%%%%%%%%%%%
%%%%%%%%%%%%%%%%%%%%%%%%%%%%%%%%%%%%%%%%%%%%%%%%%%%%%%%%%%%%%%%%%%%%%%%%%
%%%%%%%%%%%%%%%%%%%%%%%%%%%%%%%%%%%%%%%%%%%%%%%%%%%%%%%%%%%%%%%%%%%%%%%%%
\section{IEEE-754 Floating Point Number Representation}
\label{chapter::floatingpoint}

This section provides an overview of the IEEE-754 32-bit binary floating 
point format.

\begin{itemize}
\item Recall that the place values for integer binary numbers are:
\begin{verbatim}
   ... 128 64 32 16 8 4 2 1
\end{verbatim}
\item We can extend this to the right in binary similar to the way we do for 
decimal numbers:
\begin{verbatim}
   ... 128 64 32 16 8 4 2 1 . 1/2 1/4 1/8 1/16 1/32 1/64 1/128 ...
\end{verbatim}
The `.' in a binary number is a binary point, not a decimal point.

\item We use scientific notation as in $2.7 \times 10^{-47}$ to express either 
small fractions or large numbers when we are not concerned every last digit 
needed to represent the entire, exact, value of a number.

\item The format of a number in scientific notation is $mantissa \times base^{exponent}$

\item In binary we have $mantissa \times 2^{exponent}$

\item IEEE--754 format requires binary numbers to be {\em normalized} to 
$1.significand \times 2^{exponent}$ where the {\em significand}
is the portion of the {\em mantissa} that is to the right of the binary--point.

\begin{itemize}
\item The unnormalized binary value of $-2.625$ is $10.101$
\item The normalized value of $-2.625$ is $1.0101 \times 2^1$
\end{itemize}

\item We need not store the `1.' because {\em all} normalized floating 
point numbers will start that way.  Thus we can save memory when storing
normalized values by adding 1 to the significand.

{
\small
\setlength{\unitlength}{.15in}
\begin{picture}(32,4)(0,0)
	\put(0,1){\line(1,0){32}}		% bottom line
	\put(0,2){\line(1,0){32}}		% top line

	\put(0,1){\line(0,1){2}}		% left vertical
	\put(0,2){\makebox(1,1){\tiny 31}}	% left end bit number marker 

	\put(32,1){\line(0,1){2}}		% vertical right end 
	\put(31,2){\makebox(1,1){\tiny 0}}	% right end bit number marker

	\put(0,0){\makebox(1,1){\small sign}}
	\put(1,0){\makebox(8,1){\small exponent}}
	\put(9,0){\makebox(23,1){\small significand}}

    \put(0,1){\makebox(1,1){1}}		% sign

	\put(1,1){\line(0,1){2}}		% seperator
	\put(1,2){\makebox(1,1){\tiny 30}}	% bit marker

    \put(1,1){\makebox(1,1){1}}		% exponent
    \put(2,1){\makebox(1,1){0}}
    \put(3,1){\makebox(1,1){0}}
    \put(4,1){\makebox(1,1){0}}
    \put(5,1){\makebox(1,1){0}}
    \put(6,1){\makebox(1,1){0}}
    \put(7,1){\makebox(1,1){0}}
    \put(8,1){\makebox(1,1){0}}

	\put(8,2){\makebox(1,1){\tiny 23}}	% bit marker
	\put(9,1){\line(0,1){2}}		% seperator
	\put(9,2){\makebox(1,1){\tiny 22}}	% bit marker

    \put(9,1){\makebox(1,1){0}}
    \put(10,1){\makebox(1,1){1}}
    \put(11,1){\makebox(1,1){0}}
    \put(12,1){\makebox(1,1){1}}
    \put(13,1){\makebox(1,1){0}}
    \put(14,1){\makebox(1,1){0}}
    \put(15,1){\makebox(1,1){0}}
    \put(16,1){\makebox(1,1){0}}
    \put(17,1){\makebox(1,1){0}}
    \put(18,1){\makebox(1,1){0}}
    \put(19,1){\makebox(1,1){0}}
    \put(20,1){\makebox(1,1){0}}
    \put(21,1){\makebox(1,1){0}}
    \put(22,1){\makebox(1,1){0}}
    \put(23,1){\makebox(1,1){0}}
    \put(24,1){\makebox(1,1){0}}
    \put(25,1){\makebox(1,1){0}}
    \put(26,1){\makebox(1,1){0}}
    \put(27,1){\makebox(1,1){0}}
    \put(28,1){\makebox(1,1){0}}
    \put(29,1){\makebox(1,1){0}}
    \put(30,1){\makebox(1,1){0}}
    \put(31,1){\makebox(1,1){0}}
\end{picture}
}

%\item $-((1 + \frac{1}{4} + \frac{1}{16}) \times 2^{128-127}) = -(1 \frac{5}{16} \times 2^{1}) = -(1.3125 \times 2^{1}) = -2.625$
\item $-((1 + \frac{1}{4} + \frac{1}{16}) \times 2^{128-127}) = -((1 + \frac{1}{4} + \frac{1}{16}) \times 2^1) = -(2 + \frac{1}{2} + \frac{1}{8}) = -(2 + .5 + .125) = -2.625$

\item IEEE754 formats: 

\begin{tabular}{|l|l|l|}
\hline
				& IEEE754 32--bit	& IEEE754 64--bit	\\
\hline
sign			& 1 bit				& 1 bit			\\
exponent		& 8 bits (excess--127)			& 11 bits (excess-1023)		\\
mantissa		& 23 bits			& 52 bits		\\
max exponent	& 127				& 1023			\\
min exponent	& -126				& -1022			\\
\hline
\end{tabular}

\item When the exponent is all ones, the mantissa is all zeros, and
the sign is zero, the number represents positive infinity.

\item When the exponent is all ones, the mantissa is all zeros, and
the sign is one, the number represents negative infinity.

\item Note that the binary representation of an IEEE754 number in memory
can be compared for magnitude with another one using the same logic as for
comparing two's complement signed integers because the magnitude of an 
IEEE number grows upward and downward in the same fashion as signed integers.
This is why we use excess notation and locate the significand's sign bit on
the left of the exponent.

\item Note that zero is a special case number.  Recall that a normalized
number has an implied 1--bit to the left of the significand\ldots\ which
means that there is no way to represent zero!
Zero is represented by an exponent of all--zeros and a significand of 
all--zeros.  This definition allows for a positive and a negative zero 
if we observe that the sign can be either 1 or 0.

\item On the number-line, numbers between zero and the smallest fraction in 
either direction are in the {\em \gls{underflow}} areas.
\enote{Need to add the standard lecture numberline diagram showing
where the over/under--flow areas are and why.}

\item On the number line, numbers greater than the mantissa of all--ones and the 
largest exponent allowed are in the {\em \gls{overflow}} areas.

\item Note that numbers have a higher resolution on the number--line when the 
exponent is smaller.
\end{itemize}

%%%%%%%%%%%%%%%%%%%%%%%%%%%%%%%%%%%%%%%%%%%%%%%%%%%%%%%%%%%%%%%%%%%%%%%%%%
\subsection{Floating Point Number Accuracy}
Due to the finite number of bits used to store the value of a floating point
number, it is not possible to represent every one of the infinite values
on the real number line.  The following C programs illustrate this point.

%%%%%%%%%%%%%%%%%%%%%%%%%%%%%%%%%%%%%%%%%%%%%%%%%%%%%%%%%%%%%%%%%%%%%%%%%%
\subsubsection{Powers Of Two}
Just like the integer numbers, the powers of two that have bits to represent 
them can be represented perfectly\ldots\ as can their sums (provided that the
significand requires no more than 23 bits.)

\listing{powersoftwo.c}{Precise Powers of Two} 
\listing{powersoftwo.out}{Output from {\tt powersoftwo.c}}

%%%%%%%%%%%%%%%%%%%%%%%%%%%%%%%%%%%%%%%%%%%%%%%%%%%%%%%%%%%%%%%%%%%%%%%%%%
\subsubsection{Clean Decimal Numbers}
When dealing with decimal values, you will find that they don't map simply
into binary floating point values.
% (the same holds true for binary integer numbers).  

Note how the decimal numbers are not accurately represented as they get larger.
The decimal number on line 10 of \listingRef{cleandecimal.out}
can be perfectly represented in IEEE format.  However, a problem arises in 
the 11Th loop iteration.  It is due to the fact that the
binary number can not be represented accurately in IEEE format.  Its least
significant bits were truncated in a best-effort attempt at rounding the value
off in order to fit the value into the bits provided.  This is an example of
{\em low order truncation}.  Once this happens, the value of \verb@x.f@ is
no longer as precise as it could be given more bits in which to save its value.

\listing{cleandecimal.c}{Print Clean Decimal Numbers} 
\listing{cleandecimal.out}{Output from {\tt cleandecimal.c}}

%%%%%%%%%%%%%%%%%%%%%%%%%%%%%%%%%%%%%%%%%%%%%%%%%%%%%%%%%%%%%%%%%%%%%%%%%%
\subsubsection{Accumulation of Error}
These  rounding errors can be exaggerated when the number we multiply 
the \verb@x.f@ value by is, itself, something that can not be accurately 
represented in IEEE 
form.\footnote{Applications requiring accurate decimal values, such as 
financial accounting systems, can use a packed--decimal numeric format
to avoid unexpected oddities caused by the use of binary numbers.}
\enote{In a lecture one would show that one tenth is a repeating 
non--terminating binary number that gets truncated.  This discussion 
should be reproduced here in text form.}

For example, if we multiply our \verb@x.f@ value by $\frac{1}{10}$ each time, 
we can never be accurate and we start accumulating errors immediately.

\listing{erroraccumulation.c}{Accumulation of Error} 
\listing{erroraccumulation.out}{Output from {\tt erroraccumulation.c}}

%%%%%%%%%%%%%%%%%%%%%%%%%%%%%%%%%%%%%%%%%%%%%%%%%%%%%%%%%%%%%%%%%%%%%%%%%%
\subsection{Reducing Error Accumulation} 
In order to use floating point numbers in a program without causing 
excessive rounding problems an algorithm can be redesigned such that the 
accumulation is eliminated.  
This example is similar to the previous one, but this time we recalculate the 
desired value from a known--accurate integer value.  
Some rounding errors remain present, but they can not accumulate.

\listing{errorcompensation.c}{Accumulation of Error} 
\listing{errorcompensation.out}{Output from {\tt erroraccumulation.c}}
