\chapter{Introduction}
\label{chapter:Introduction}

At its core, a digital computer has at least one \acrfull{cpu}.  A
CPU executes a continuous stream of instructions called a \gls{program}.  
These program instructions are expressed in what is called 
\gls{MachineLanguage}.  Each machine language instruction is a binary value.  
In order to provide a method to simplify the management of machine language 
programs a symbolic mapping is provided where a mneumonic can be used to 
specify each machine instruction and any of its parameters\ldots\ rather 
than mandate that programs be expressed as binary machine language 
instructions.  The set of mneumonics, parameters and rules for specifying 
them is called an {\em Assembly Language}.

%%%%%%%%%%%%%%%%%%%%%%%%%%%%%%%%%%%%%%%%%%%%%%%%%%%%%%%%%%%%%%%%%%%%%%%%%%%%%%%%
%%%%%%%%%%%%%%%%%%%%%%%%%%%%%%%%%%%%%%%%%%%%%%%%%%%%%%%%%%%%%%%%%%%%%%%%%%%%%%%%
\section{The Digital Computer}

A digital computer is composed of storage systems (memory, disc drives,
USB drives, etc.), a CPU (with one or more cores), input peripherals like 
a keyboard and mouse and output peripherals like a display or speakers.

\subsection{Storage Systems}

Computer storage systems are used to hold the data and instructions
for the CPU.

Types of computer storage can be classified into two categories.
Volatile and non--volatile.

\subsubsection{Volatile Storage}

Volatile storage is characterized by the fact that it will lose its
contents (forget) any time that it is powered off.

One type of volatile storage is provided inside the CPU itself in 
small blocks called \glspl{register}.  These registers are used to 
hold individual data values that can be manipulated by the instructions
that are executed by the CPU.  

Another type of volatile storage is main memory.
Main memory is connected to a computer's CPU and is used to hold
the data and instructions that can not fit into the CPU registers.

Typically, a CPU's registers can hold tens of data values while
the main memory can contain many billions of data values.

A CPU can process data in a register at a speed that can be an order 
of magnitude faster than the rate that it can process (specifically,
transfer data and instructions to and from) the main memory.  

Register storage costs an order of magnitude more to manufacture than
main memory.  While it is desirable to have many registers the economics 
dictate that the vast majority of volatile computer storage be provided
in its main memory.  As a result, optimizing the copying of data between 
the registers and main memory is a desirable trait of good programs.

\subsubsection{Non--Volatile Storage}

Non--volatile storage is characterized by the fact that it will {\em NOT} 
lose its contents when it is powered off.

Common types of non--volatile storage are disc drives, flash cards and USB 
drives.  Prices can vary widely depending on size and transfer speeds.

It is typical for a computer system's non--volatile storage to operate
more slowly than its main memory.

\subsection{CPU}

The \acrshort{cpu} is a collection of registers and circuitry designed
to read data and instructions from the system storage.  The instructions
are used to instruct the CPU how to perform various mathamatical and 
logical operations on the data in its registers and write the results
of those operations back into the system storage.

\subsection{Peripherals}

A peripheral is a device that is not a CPU or main memory.  They are 
typically used to transfer information/data into and out of the 
main memory.

This text is not particularly concerned with the peripherals of a computer
system other than in those sections where instructions are discussed 
whose purpose is to address the needs of a peripheral device.  Such
instructions are used to initiate, execute and/or synchronize data transfers.


%%%%%%%%%%%%%%%%%%%%%%%%%%%%%%%%%%%%%%%%%%%%%%%%%%%%%%%%%%%%%%%%%%%%%%%%%%%%%%%%
%%%%%%%%%%%%%%%%%%%%%%%%%%%%%%%%%%%%%%%%%%%%%%%%%%%%%%%%%%%%%%%%%%%%%%%%%%%%%%%%
\section{Instruction Set Architecture}

The catalog of rules that describe all of the details of the instructions 
that a given CPU can execute is called its \acrfull{isa}.

The RISC--V CPU ISA is defined as a set of modules.  The purpose of
dividing the ISA into modules is to allow an implementor to select which 
features to incorporate into a CPU design.

Any given RISC--V implementation must provide one of the {\em base}
modules and zero or more of the {\em extension} modules.

\subsection{RV Base Modules}
The base modules are RV32I (32--bit general purpose), 
RV32E (32--bit embedded), RV64I (64--bit general purpose) 
and RV128I (128--bit general purpose).

These base modules provide the minimal functional set of integer operations
needed to execute an application.  The differing bit--widths address
the needs of different main--memory sizes.


\subsection{Extension Modules}

RISC-V extension modules may be included by an implementor interested
in optimizing a design for one or more purposes.

Available extension modules include M (integer math), A (atomic),
F (32--bit floating point), D (64--bit floating point), 
Q (128--bit floating point), C (compressed size instructions) and others.

The extension name {\em G} is used to represent the combined set of IMAFD
extensions as is expected to be a common combination.


This text discusses programming the RV32IM ISA using assembly language. 
