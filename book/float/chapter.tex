\chapter{Floating Point Numbers}
\label{chapter:floatingpoint}

%%%%%%%%%%%%%%%%%%%%%%%%%%%%%%%%%%%%%%%%%%%%%%%%%%%%%%%%%%%%%%%%%%%%%%%%%
%%%%%%%%%%%%%%%%%%%%%%%%%%%%%%%%%%%%%%%%%%%%%%%%%%%%%%%%%%%%%%%%%%%%%%%%%
%%%%%%%%%%%%%%%%%%%%%%%%%%%%%%%%%%%%%%%%%%%%%%%%%%%%%%%%%%%%%%%%%%%%%%%%%
\section{IEEE-754 Floating Point Number Representation}
\label{chapter::floatingpoint}

This section provides an overview of the IEEE-754 32-bit binary floating 
point format.

\begin{itemize}
\item Recall that the place values for integer binary numbers are:
\begin{verbatim}
   ... 128 64 32 16 8 4 2 1
\end{verbatim}
\item We can extend this to the right in binary similar to the way we do for 
decimal numbers:
\begin{verbatim}
   ... 128 64 32 16 8 4 2 1 . 1/2 1/4 1/8 1/16 1/32 1/64 1/128 ...
\end{verbatim}
The `.' in a binary number is a binary point, not a decimal point.

\item We use scientific notation as in $2.7 \times 10^{-47}$ to express either 
small fractions or large numbers when we are not concerned every last digit 
needed to represent the entire, exact, value of a number.

\item The format of a number in scientific notation is $mantissa \times base^{exponent}$

\item In binary we have $mantissa \times 2^{exponent}$

\item IEEE-754 format requires binary numbers to be {\em normalized} to 
$1.significand \times 2^{exponent}$ where the {\em significand}
is the portion of the {\em mantissa} that is to the right of the binary-point.

\begin{itemize}
\item The unnormalized binary value of $-2.625$ is $-10.101$
\item The normalized value of $-2.625$ is $-1.0101 \times 2^1$
\end{itemize}

\item We need not store the `1.' part because {\em all} normalized floating 
point numbers will start that way.  Thus we can save memory when storing
normalized values by inserting a `1.' to the left of significand.

{
\small
\setlength{\unitlength}{.15in}
\begin{picture}(32,4)(0,0)
	\put(0,1){\line(1,0){32}}		% bottom line
	\put(0,2){\line(1,0){32}}		% top line

	\put(0,1){\line(0,1){2}}		% left vertical
	\put(0,2){\makebox(1,1){\tiny 31}}	% left end bit number marker 

	\put(32,1){\line(0,1){2}}		% vertical right end 
	\put(31,2){\makebox(1,1){\tiny 0}}	% right end bit number marker

	\put(0,0){\makebox(1,1){\small sign}}
	\put(1,0){\makebox(8,1){\small exponent}}
	\put(9,0){\makebox(23,1){\small significand}}

    \put(0,1){\makebox(1,1){1}}		% sign

	\put(1,1){\line(0,1){2}}		% seperator
	\put(1,2){\makebox(1,1){\tiny 30}}	% bit marker

    \put(1,1){\makebox(1,1){1}}		% exponent
    \put(2,1){\makebox(1,1){0}}
    \put(3,1){\makebox(1,1){0}}
    \put(4,1){\makebox(1,1){0}}
    \put(5,1){\makebox(1,1){0}}
    \put(6,1){\makebox(1,1){0}}
    \put(7,1){\makebox(1,1){0}}
    \put(8,1){\makebox(1,1){0}}

	\put(8,2){\makebox(1,1){\tiny 23}}	% bit marker
	\put(9,1){\line(0,1){2}}		% seperator
	\put(9,2){\makebox(1,1){\tiny 22}}	% bit marker

    \put(9,1){\makebox(1,1){0}}
    \put(10,1){\makebox(1,1){1}}
    \put(11,1){\makebox(1,1){0}}
    \put(12,1){\makebox(1,1){1}}
    \put(13,1){\makebox(1,1){0}}
    \put(14,1){\makebox(1,1){0}}
    \put(15,1){\makebox(1,1){0}}
    \put(16,1){\makebox(1,1){0}}
    \put(17,1){\makebox(1,1){0}}
    \put(18,1){\makebox(1,1){0}}
    \put(19,1){\makebox(1,1){0}}
    \put(20,1){\makebox(1,1){0}}
    \put(21,1){\makebox(1,1){0}}
    \put(22,1){\makebox(1,1){0}}
    \put(23,1){\makebox(1,1){0}}
    \put(24,1){\makebox(1,1){0}}
    \put(25,1){\makebox(1,1){0}}
    \put(26,1){\makebox(1,1){0}}
    \put(27,1){\makebox(1,1){0}}
    \put(28,1){\makebox(1,1){0}}
    \put(29,1){\makebox(1,1){0}}
    \put(30,1){\makebox(1,1){0}}
    \put(31,1){\makebox(1,1){0}}
\end{picture}
}

%\item $-((1 + \frac{1}{4} + \frac{1}{16}) \times 2^{128-127}) = -(1 \frac{5}{16} \times 2^{1}) = -(1.3125 \times 2^{1}) = -2.625$
\item $-((1 + \frac{1}{4} + \frac{1}{16}) \times 2^{128-127}) = -((1 + \frac{1}{4} + \frac{1}{16}) \times 2^1) = -(2 + \frac{1}{2} + \frac{1}{8}) = -(2 + .5 + .125) = -2.625$

\item IEEE-754 formats: 

\begin{tabular}{|l|l|l|}
\hline
				& IEEE-754 32-bit	& IEEE-754 64-bit	\\
\hline
sign			& 1 bit				& 1 bit			\\
exponent		& 8 bits (excess-127)			& 11 bits (excess-1023)		\\
mantissa		& 23 bits			& 52 bits		\\
max exponent	& 127				& 1023			\\
min exponent	& -126				& -1022			\\
\hline
\end{tabular}

\item When the exponent is all ones, the mantissa is all zeros, and
the sign is zero, the number represents positive infinity.

\item When the exponent is all ones, the mantissa is all zeros, and
the sign is one, the number represents negative infinity.

\item Note that the binary representation of an IEEE-754 number in memory
can be compared for magnitude with another one using the same logic as for
comparing two's complement signed integers because the magnitude of an 
IEEE number grows upward and downward in the same fashion as signed integers.
This is why we use excess notation and locate the significand's sign bit on
the left of the exponent.

\item Note that zero is a special case number.  Recall that a normalized
number has an implied 1-bit to the left of the significand\ldots\ which
means that there is no way to represent zero!
Zero is represented by an exponent of all-zeros and a significand of 
all-zeros.  This definition allows for a positive and a negative zero 
if we observe that the sign can be either 1 or 0.

\item On the number-line, numbers between zero and the smallest fraction in 
either direction are in the {\em \gls{underflow}} areas.
\enote{Need to add the standard lecture number-line diagram showing
where the over/under-flow areas are and why.}

\item On the number line, numbers greater than the mantissa of all-ones and the 
largest exponent allowed are in the {\em \gls{overflow}} areas.

\item Note that numbers have a higher resolution on the number line when the 
exponent is smaller.
\end{itemize}

%%%%%%%%%%%%%%%%%%%%%%%%%%%%%%%%%%%%%%%%%%%%%%%%%%%%%%%%%%%%%%%%%%%%%%%%%%
\subsection{Floating Point Number Accuracy}
Due to the finite number of bits used to store the value of a floating point
number, it is not possible to represent every one of the infinite values
on the real number line.  The following C programs illustrate this point.

%%%%%%%%%%%%%%%%%%%%%%%%%%%%%%%%%%%%%%%%%%%%%%%%%%%%%%%%%%%%%%%%%%%%%%%%%%
\subsubsection{Powers Of Two}
Just like the integer numbers, the powers of two that have bits to represent 
them can be represented perfectly\ldots\ as can their sums (provided that the
significand requires no more than 23 bits.)

\listing{powersoftwo.c}{Precise Powers of Two} 
\listing{powersoftwo.out}{Output from {\tt powersoftwo.c}}

%%%%%%%%%%%%%%%%%%%%%%%%%%%%%%%%%%%%%%%%%%%%%%%%%%%%%%%%%%%%%%%%%%%%%%%%%%
\subsubsection{Clean Decimal Numbers}
When dealing with decimal values, you will find that they don't map simply
into binary floating point values.
% (the same holds true for binary integer numbers).  

Note how the decimal numbers are not accurately represented as they get larger.
The decimal number on line 10 of \listingRef{cleandecimal.out}
can be perfectly represented in IEEE format.  However, a problem arises in 
the 11Th loop iteration.  It is due to the fact that the
binary number can not be represented accurately in IEEE format.  Its least
significant bits were truncated in a best-effort attempt at rounding the value
off in order to fit the value into the bits provided.  This is an example of
{\em low order truncation}.  Once this happens, the value of \verb@x.f@ is
no longer as precise as it could be given more bits in which to save its value.

\listing{cleandecimal.c}{Print Clean Decimal Numbers} 
\listing{cleandecimal.out}{Output from {\tt cleandecimal.c}}

%%%%%%%%%%%%%%%%%%%%%%%%%%%%%%%%%%%%%%%%%%%%%%%%%%%%%%%%%%%%%%%%%%%%%%%%%%
\subsubsection{Accumulation of Error}
These  rounding errors can be exaggerated when the number we multiply 
the \verb@x.f@ value by is, itself, something that can not be accurately 
represented in IEEE 
form.\footnote{Applications requiring accurate decimal values, such as 
financial accounting systems, can use a packed-decimal numeric format
to avoid unexpected oddities caused by the use of binary numbers.}
\enote{In a lecture one would show that one tenth is a repeating 
non-terminating binary number that gets truncated.  This discussion 
should be reproduced here in text form.}

For example, if we multiply our \verb@x.f@ value by $\frac{1}{10}$ each time, 
we can never be accurate and we start accumulating errors immediately.

\listing{erroraccumulation.c}{Accumulation of Error} 
\listing{erroraccumulation.out}{Output from {\tt erroraccumulation.c}}

%%%%%%%%%%%%%%%%%%%%%%%%%%%%%%%%%%%%%%%%%%%%%%%%%%%%%%%%%%%%%%%%%%%%%%%%%%
\subsection{Reducing Error Accumulation} 
In order to use floating point numbers in a program without causing 
excessive rounding problems an algorithm can be redesigned such that the 
accumulation is eliminated.  
This example is similar to the previous one, but this time we recalculate the 
desired value from a known-accurate integer value.  
Some rounding errors remain present, but they can not accumulate.

\listing{errorcompensation.c}{Accumulation of Error} 
\listing{errorcompensation.out}{Output from {\tt erroraccumulation.c}}
