
\oddsidemargin=.82in
%\evensidemargin=1.82in
\evensidemargin=1.72in

%\topmargin=-.5in
\headheight=12pt
\headsep=20pt

%\hoffset=-1.2in      % for 8.5x11
\hoffset=-1.1in      % for 8.5x11

%\voffset=-1.15in        % for 8.5x11
\voffset=-.8in        % for 8.5x11

%\textheight=9.75in   % for 8.5x11
\textheight=9in   % for 8.5x11
%\textheight=8.75in   % for 8.5x11

\textwidth=6.1in    % for 8.5x11
%\textwidth=6.25in    % for 8.5x11

\marginparsep=7pt
%\marginparwidth=71pt
%\marginparwidth=1.25in
%\marginparwidth=1.5in
\marginparwidth=1.25in
\footskip=36pt
\marginparpush=5pt


\usepackage{ifthen}
\usepackage{stringstrings}	% so can count characters in a string
\usepackage{xstring}		% so can count characters in a string

\usepackage[pass]{geometry}
\usepackage{color}   % Necessary for colored links

% load makeidx BEFORE hyperref to make the index clickable
% load makeidx AFTER hyperref if want to use showidx
\usepackage{makeidx}
\usepackage[pagebackref]{hyperref}
\hypersetup{
    colorlinks=true,	%set true if you want colored links
    linkcolor=blue		%choose some color if you want links to stand out
}

% don't say subsection and subsubsection in auto-references
\let\subsectionautorefname\sectionautorefname
\let\subsubsectionautorefname\sectionautorefname

%\usepackage{makeidx,showidx}	% showidx breaks hyperref when loaded before hyperref


%%%%%%%%%%%%%%%%%%%%%%%%%%%%%%%%%%%%%%%%%%%%%%%%%%%%%%%%%%%%%%%%%%%%%%
% for one-sided on left margin
%\usepackage{lineno}
%\linenumbers

% for two-sided on inside margin
%\usepackage[switch*,pagewise]{lineno}
%\linenumbers
%\runningpagewiselinenumbers

%\renewcommand\linenumberfont{\normalfont\tiny\sffamily\bfseries\color{violet}}

% lineno is screwy for displayed equations
%\let\oldequation\equation
%\let\oldendequation\endequation
%\renewenvironment{equation}
%  {\linenomathNonumbers\oldequation}
%  {\oldendequation\endlinenomath}
%
%\setlength\linenumbersep{6mm}
%%%%%%%%%%%%%%%%%%%%%%%%%%%%%%%%%%%%%%%%%%%%%%%%%%%%%%%%%%%%%%%%%%%%%%


%\usepackage{url}

\usepackage{lastpage}
\usepackage{fancyhdr}

\usepackage{amsmath}
\numberwithin{equation}{section}
\usepackage{amsfonts} % I hear these are also good to load
\usepackage{amssymb} % I hear these are also good to load

%\usepackage{picture}
%\usepackage{epstopdf}
%\usepackage{graphicx}
\usepackage{epsfig}
\usepackage{tikz-timing}
\usepackage{tikz}
%\usepackage{timing}
\usepackage{float}
\usepackage{fancyvrb}

%\usepackage{caption}
\usepackage[hypcap=true]{caption}		% point to top of figure rather than caption in \hyperref

\usepackage{placeins}

\usepackage{listings}

\usepackage[toc]{glossaries}
%\renewcommand*{\glsclearpage}{} 

\usepackage{pifont}
\usepackage{layout}

\usepackage{xcolor}          
\usepackage{textcomp}       % for the trademark symbol

%\usepackage[obeyspaces]{url}
\usepackage{fink}			% deprecated in favor of currfile
%\usepackage{currfile}		% dut... doesn't emit the local include path properly

\usepackage{MyFigs}

\def\code#1{\url{#1}}

% The exercise environment
\usepackage{exercise}

\renewcommand{\ExerciseHeader}{%
\textbf{\large\ExerciseHeaderDifficulty\ExerciseName\ %
\ExerciseHeaderNB\ExerciseHeaderTitle\ExerciseHeaderOrigin}\medskip}

\renewcommand{\ExePartHeader}{%
\medskip\emph{\large\ExePartHeaderDifficulty\ExePartName %
\ExePartHeaderNB \ExePartHeaderTitle\\
}}

%\renewcommand{\ExePartHeader}{%
%\medskip\emph{\large\ExePartHeaderDifficulty Part \ExePartHeaderNB %
%\quad \ExePartName\ExePartHeaderTitle}}



%%%%%%%%%%%%%%%%%%%%%%%%%%%%%%%%%%%%%%%%%%%%%%%%%%%%%%%
%\DefineVerbatimEnvironment%
%{Code}{Verbatim}
%{frame=single,numbers=left,numbersep=2mm,framesep=3mm}
%%%,numbersep=2mm,frame=lines,framerule=0.8mm,framesep=5mm

%%%%%%%%%%%%%%%%%%%%%%%%%%%%%%%%%%%%%%%%%%%%%%%%%%%%%%%
% [1] = filename to include
% [2] = title of the code sample
% A label will be generated as: Code:#1
%\newcommand{\xxxx}[2]{%
%	\label{Code:#1} %
%	\VerbatimInput[frame=single,numbers=left,numbersep=2mm,framesep=3mm,label={#2}]{#1}}

%\newcommand{\theListingFontFamily}{\ttfamily\small}
\newcommand{\theListingFontFamily}{\ttfamily\footnotesize}
%\newcommand{\theListingFontFamily}{\ttfamily\scriptsize}

%%%%%%%%%%%%%%%%%%%%%%%%%%%%%%%%%%%%%%%%%%%%%%%%%%%%%%%
% [1] = place for additional listing parameters, default: language=C
% [2] = filepath
% [3] = Description of the listing
%%%%%%%%%%%%%%%%%%%%%%%%%%%%%%%%%%%%%%%%%%%%%%%%%%%%%%%
\newcommand{\listing}[3][language=C]{%
	\lstinputlisting[ %
		numbers=left,numberstyle=\tiny,stepnumber=1,numbersep=8pt, %
		breaklines=true, %
		frame=single %
		showtabs=false, %
		basicstyle=\theListingFontFamily, %
		showstringspaces=false, %
		tabsize=4, %
		showlines=true, %  show all blank lines
		#1, %
		captionpos=t,frame=tblr,label={lst:\detokenize{#2}},caption={{\tt \detokenize{#2}}\\\hspace{\textwidth}{\small #3}}]{\detokenize{#2}}}

%%%%%%%%%%%%%%%%%%%%%%%%%%%%%%%%%%%%%%%%%%%%%%%%%%%%%%%
\newcommand{\listingRef}[1]{\autoref{lst:\detokenize{#1}}}

%%%%%%%%%%%%%%%%%%%%%%%%%%%%%%%%%%%%%%%%%%%%%%%%%%%%%%%
\lstnewenvironment{tty}
	{\lstset{language=sh, %
		numbers=left,numberstyle=\tiny,stepnumber=1,numbersep=8pt, %
		breaklines=true, %
		frame=single, %
		showtabs=false, %
		basicstyle=\theListingFontFamily, %
		showstringspaces=false, %
		tabsize=4, %
		showlines=true, %  show all blank lines
		basicstyle=\theListingFontFamily}}
	{}


%%%%%%%%%%%%%%%%%%%%%%%%%%%%%%%%%%%%%%%%%%%%%%%%%%%%%%%
% A footer that shall appear on every page

%\newcommand{\MyFoot}{{\sf\scriptsize Copyright \copyright\ 2014, 2015 John Winans.  All Rights Reserved}\\
%\vspace{.05in}
%\scriptsize\FooterText}

\newcommand{\MyFoot}{\scriptsize\FooterText}

%%%%%%%%%%%%%%%%%%%%%%%%%%%%%%%%%%%%%%%%%%%%%%%%%%%%%%%
%%%%%%%%%%%%%%%%%%%%%%%%%%%%%%%%%%%%%%%%%%%%%%%%%%%%%%%
\pagestyle{fancy}

% supress normal headings and footers
\fancyhf{}

% heading and footing rules
\renewcommand{\headrulewidth}{1pt}
\renewcommand{\footrulewidth}{1pt}


%\def\DiscardOneCharacter#1{}
% Sub-footer that shows the version control version string in the lfoot defined above
\ifdefined\GitFileName
%    \newcommand{\FooterText}{\tt \GitFileName \currfilename\\
    \newcommand{\FooterText}{\tt \GitFileName \finkpath\\
\GitDescription}
\else
    \newcommand{\FooterText}{\emph{--UNKNOWN--}}
\fi

%\lhead{\leftmark}
%\rhead{\rightmark}
\fancyhead[LE]{\leftmark}
\fancyhead[RO]{\rightmark}

\newcommand{\PageNumber}{Page \thepage\ of \pageref*{LastPage}}
%\rfoot{Page \thepage\ of \pageref{LastPage}}
%\lfoot{\MyFoot}

\fancyfoot[LE,RO]{\PageNumber}
\fancyfoot[RE,LO]{\MyFoot}


%%%%%%%%%%%%%%%%%%%%%%%%%%%%%%%%%%%%%%%%%%%%%%%%%%%%%%%
% override the plain page style so the first page of a
% chapter still has a footer on it (but no header).

\fancypagestyle{plain}{%
\renewcommand{\headrulewidth}{0pt} %
\fancyhf{} % clear all header and footer fields
\fancyfoot[LE,RO]{\PageNumber}
\fancyfoot[RE,LO]{\MyFoot}
%\lfoot{\MyFoot} %
%\rfoot{Page \thepage\ of \pageref*{LastPage}}
}

%%%%%%%%%%%%%%%%%%%%%%%%%%%%%%%%%%%%%%%%%%%%%%%%%%%%%%%

\setlength{\parindent}{0pt}
\setlength{\parskip}{.51em}

% How deep should we enumerate the section/subsection/subsubsections
% 3=all the way
\setcounter{secnumdepth}{3}

% How many section-levels to show in the TOC.
% 4=all of them
%\setcounter{tocdepth}{4}
\setcounter{tocdepth}{1}



%%%%%%%%%%%%%%%%%%%%%%%%%%%%%%%%%%%%%%%%%%%%%%%%%%%%%%%
% Presentation styles for things like names links and
% window objects in figures.

% how should we present a window name
\newcommand{\windowname}[1]{{\em #1}} 

% How should we present the name of an object in a figure
% that can be interacted with.
\newcommand{\linkname}[1]{{\bf #1}}


%%%%%%%%%%%%%%%%%%%%%%%%%%%%%%%%%%%%%%%%%%%%%%%%%%%%%%%
% fix some annoying things

\newcommand{\tm}{\textsuperscript{TM}}
\newcommand{\rtm}{\textsuperscript{\textregistered}}


%%%%%%%%%%%%%%%%%%%%%%%%%%%%%%%%%%%%%%%%%%%%%%%%%%%%%%%
% A command to make a colored background box whilst in math mode:
\newcommand{\MathHilight}[1]{\colorbox{yellow}{\ensuremath{#1}}}

\usepackage{tcolorbox}
%\tcbset{colback=blue!20!white}
\tcbset{colback=green!20!white}

