\newglossaryentry{latex}
{
    name=LaTeX,
    description={Is a mark up language specially suited
    for scientific documents}
}

\newglossaryentry{binary}
{
	name=binary,
	description={Something that has two parts or states.  In computing
		these two states are represented by the numbers one and zero or
		by the conditions true and false and can be stored in one bit}
}
\newglossaryentry{hexadecimal}
{
	name=hexadecimal,
	description={A base--16 numbering system whose digits are 
		0123456789abcdef.  The hex digits (hits) are not case--sensitive}
}
\newglossaryentry{bit}
{
	name=bit,
	description={One binary digit}
}
\newglossaryentry{hit}
{
	name={hit},
	description={One hex digit}
}
\newglossaryentry{byte}
{
	name=byte,
	description={A binary value represented by 8 bits}
}
\newglossaryentry{HalfWord}
{
	name={Halfword},
	description={A binary value represented by 16 bits}
}
\newglossaryentry{FullWord}
{
	name={Fullword},
	description={A binary value represented by 32 bits}
}
\newglossaryentry{DoubleWord}
{
	name={Doubleword},
	description={A binary value represented by 64 bits}
}
\newglossaryentry{QuadWord}
{
	name={Quadword},
	description={A binary value represented by 128 bits}
}
\newglossaryentry{HighOrderBits}
{
	name={High order bits},
	description={Some number of MSBs}
}
\newglossaryentry{LowOrderBits}
{
    name={Low order bits},
    description={Some number of LSBs}
}

\newglossaryentry{xlen}
{
	name=XLEN,
	description={The number of bits a RISC-V x integer register 
		(such as x0).  For RV32 XLEN=32, RV64 XLEN=64 etc}
}
\newglossaryentry{rv32}
{
	name=RV32,
	description={Short for RISC-V 32.  The number 32 refers to the XLEN}
}
\newglossaryentry{rv64}
{
	name=RV64,
	description={Short for RISC-V 64.  The number 64 refers to the XLEN}
}
\newglossaryentry{overflow}
{
	name=overflow,
	description={The situation where the result of an addition or 
		subtraction operation is approaching positive or negative 
		infinity and exceeds the number of bits alloted to contain 
		the result.  This is typically caused by high--order truncation}
}
\newglossaryentry{underflow}
{
	name=underflow,
	description={The situation where the result of an addition or 
		subtraction operation is approaching zero and exceeds the number 
		of bits alloted to contain the result.  This is typically
        caused by low--order truncation}
}

\newglossaryentry{MachineLanguage}
{
	name={machine language},
	description={The instructions that are executed by a CPU that are expressed
		in the form of binary values}
}
\newglossaryentry{register}
{
	name={register},
	description={A unit of storage inside a CPU with the capacity of XLEN bits}
}
\newglossaryentry{program}
{
	name={program},
	description={A ordered list of one or more instructions}
}
\newglossaryentry{address}
{
	name={address},
	description={A numeric value used to uniquely identify each byte of main memory}
}
\newglossaryentry{alignment}
{
	name={alignment},
	description={Refers to a range of numeric values that begin 
		at a multiple of some number.  Primairly used when referring to
		a memory address.  For example an alignment of two refers to one
		or more addresses starting at even address and continuing onto
		subsequent adjacent, increasing memory addresses}
}
\newglossaryentry{exception}
{
	name={exception},
	description={An error encountered by the CPU while executing an instruction
		that can not be completed}
}

\newglossaryentry{bigendian}
{
	name={big endian},
	description={A number format where the most significant values are 
	printed to the left of the lesser significant values.  This is the
	method that everyone used to write decimal numbers every day}
}
\newglossaryentry{littleendian}
{
	name={little endian},
	description={A number format where the least significant values are 
		printed to the left of the more significant values.  This is the
		opposite ordering that everyone learns in grade school when learning
		how to count.  For example a big endian number written as ``1234''
		would be written in little endian form as ``4321''}
}
\newglossaryentry{rvddt}
{
	name={rvddt},
	description={A RV32I simulator and debugging tool inspired by the 
		simplicity of the Dynamic Debugging Tool (ddt) that was part of 
		the CP/M operating system}
}
\newglossaryentry{mneumonic}
{
	name={mneumonic},
	description={A method used to remember something.  In the case of
		assembly language, each machine instruction is given a name
		so the programmer need not memorize the biary values of each
		machine instruction}
}

\newacronym{hart}{hart}{Hardware Thread}
\newacronym{msb}{MSB}{Most Significant Bit}
\newacronym{lsb}{LSB}{Least Significant Bit}
\newacronym{isa}{ISA}{Instruction Set Architecture}
\newacronym{cpu}{CPU}{Central Processing Unit}
