\newglossaryentry{latex}
{
    name=LaTeX,
    description={Is a mark up language specially suited
    for scientific documents}
}

\newglossaryentry{binary}
{
	name=binary,
	description={Something that has two parts or states.  In computing
		these two states are represented by the numbers one and zero or
		by the conditions true and false and can be stored in one bit}
}

\newglossaryentry{bit}
{
	name=bit,
	description={One binary digit}
}
\newglossaryentry{hit}
{
	name={hit},
	description={One hex digit}
}
\newglossaryentry{byte}
{
	name=byte,
	description={A binary value represented by 8 bits}
}
\newglossaryentry{HalfWord}
{
	name={Halfword},
	description={A binary value represented by 16 bits}
}
\newglossaryentry{FullWord}
{
	name={Fullword},
	description={A binary value represented by 32 bits}
}
\newglossaryentry{DoubleWord}
{
	name={Doubleword},
	description={A binary value represented by 64 bits}
}
\newglossaryentry{QuadWord}
{
	name={Quadword},
	description={A binary value represented by 128 bits}
}
\newglossaryentry{HighOrderBits}
{
	name={High order bits},
	description={Some number of MSBs}
}
\newglossaryentry{LowOrderBits}
{
    name={Low order bits},
    description={Some number of LSBs}
}

\newglossaryentry{xlen}
{
	name=XLEN,
	description={The number of bits a RISC-V x integer register 
		(such as x0).  For RV32 XLEN=32, RV64 XLEN=64 etc}
}
\newglossaryentry{rv32}
{
	name=RV32,
	description={Short for RISC-V 32.  The number 32 refers to the XLEN}
}
\newglossaryentry{rv64}
{
	name=RV64,
	description={Short for RISC-V 64.  The number 64 refers to the XLEN}
}
\newglossaryentry{overflow}
{
	name=overflow,
	description={The situation where the result of an addition or 
		subtraction operation is approaching positive or negative 
		infinity and exceeds the number of bits alloted to contain 
		the result.  This is typically caused by high--order truncation}
}
\newglossaryentry{underflow}
{
	name=underflow,
	description={The situation where the result of an addition or 
		subtraction operation is approaching zero and exceeds the number 
		of bits alloted to contain the result.  This is typically
        caused by low--order truncation}
}

\newglossaryentry{MachineLanguage}
{
	name={machine language},
	description={The instructions that are executed by a CPU that are expressed
		in the form of binary values}
}
\newglossaryentry{register}
{
	name={register},
	description={A unit of storage inside a CPU}
}
\newglossaryentry{program}
{
	name={program},
	description={A ordered list of one or more instructions}
}





\newacronym{msb}{MSB}{Most Significant Bit}
\newacronym{lsb}{LSB}{Least Significant Bit}
\newacronym{isa}{ISA}{Instruction Set Architecture}
\newacronym{cpu}{CPU}{Central Processing Unit}
