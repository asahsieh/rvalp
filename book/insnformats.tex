
\def\SignBoxCornerRadius{.75mm}

%%%%%%%%%%%%%%%%%%%%%%%%%%%%%%%%%%%%%%%%%%%%%%%%%%%%%%%%%%%%%
\newcommand\BeginTikzPicture{
    \begin{tikzpicture}[x=.4cm,y=.3cm]
}

%%%%%%%%%%%%%%%%%%%%%%%%%%%%%%%%%%%%%%%%%%%%%%%%%%%%%%%%%%%%%
\newcommand\EndTikzPicture{
    \end{tikzpicture}
}

%%%%%%%%%%%%%%%%%%%%%%%%%%%%%%%%%%%%%%%%%%%%%%%%%%%%%%%%%%%%%
% Print the characters within a string evenly spaced at integral node positions
%
% #1 The number of characters in the string
\newcommand\DrawBitstring[2]{
\foreach \x in {1,...,#1}%
   	\draw(\x,0) node{\substring{#2}{\x}{\x}};%
}

%%%%%%%%%%%%%%%%%%%%%%%%%%%%%%%%%%%%%%%%%%%%%%%%%%%%%%%%%%%%%
% #1 The total size
% #2 The string to print
% #3 The value to use when extending to left
\newcommand\DrawLeftExtendedBitstring[3]{
	\StrLen{#2}[\numchars]

	\pgfmathsetmacro\leftpadd{int(#1-\numchars)}
	\foreach \x in {1,...,\leftpadd}
    	\draw(\x,0) node{#3};

	\pgfmathsetmacro\leftpadd{int(\leftpadd+1)}
	\foreach \x in {\leftpadd,...,#1}
		\pgfmathsetmacro\ix{int(\x-\leftpadd+1)}
		\draw(\x,0) node{\substring{#2}{\ix}{\ix}};
}

%%%%%%%%%%%%%%%%%%%%%%%%%%%%%%%%%%%%%%%%%%%%%%%%%%%%%%%%%%%%%
% If the string is shorter than expected, extend with #5 to the right.
%
% #1 The total size
% #2 Num chars to extend on the right
% #3 The string to print
% #4 The value to use when extending to left
% #5 The value to use when extending to right
\newcommand\DrawDoubleExtendedBitstring[5]{
	\StrLen{#3}[\numchars]

	\pgfmathsetmacro\leftpadd{int(#1-#2-\numchars)}
	\ifthenelse{1 > \leftpadd}
	{}
	{
		\foreach \x in {1,...,\leftpadd}
    		\draw(\x,0) node{#4};
	}

	\pgfmathsetmacro\leftpadd{int(\leftpadd+1)}
	\pgfmathsetmacro\rightpadd{int(\leftpadd+\numchars)}
	\foreach \x in {\leftpadd,...,\rightpadd}
		\pgfmathsetmacro\ix{int(\x-\leftpadd+1)}
		\draw(\x,0) node{\substring{#3}{\ix}{\ix}};


	%\pgfmathsetmacro\rightpadd{int(\rightpadd+1)}
	\ifthenelse{\rightpadd > #1}	
	{}
	{
		\foreach \x in {\rightpadd,...,#1}
			\draw(\x,0) node{#5};
	}
}




%%%%%%%%%%%%%%%%%%%%%%%%%%%%%%%%%%%%%%%%%%%%%%%%%%%%%%%%%%%%%
% Draw a box suitable to show the given number of bits in a 
% labeled box suitable for showing expanded binary numbers.
%
% #1 The number of characters to display
\newcommand\DrawBitBox[1]{
    \draw (.5,-.75) -- (#1+.5,-.75);		% box bottom
    \draw (.5,.75) -- (#1+.5,.75);			% box top
    \draw (.5,-.75) -- (.5, 1.5);			% left end
    \draw (#1+.5,-.75) -- (#1+.5, 1.5);		% right end
    \pgfmathsetmacro\result{int(#1-1)}		% calc high bit 
    \node at (1,1.2) {\tiny\result};		% high bit label
    \draw(#1,1.2) node{\tiny0};				% low bit label

    \pgfmathsetmacro\result{#1/2}
    \node at (\result,-1.2) {\tiny#1};		% size below the box

    \pgfmathsetmacro\result{#1/2}
    \draw[->] (\result+.6,-1.2) -- (#1+.5,-1.2);
    \draw[->] (\result-.6,-1.2) -- (.5,-1.2);
}


%%%%%%%%%%%%%%%%%%%%%%%%%%%%%%%%%%%%%%%%%%%%%%%%%%%%%%%%%%%%%
\newcommand\DrawBitBoxUnsigned[1]{
	\StrLen{#1}[\numchars]
	\DrawBitBox{\numchars}
	\DrawBitstring{\numchars}{#1}		% show the bits
}



%%%%%%%%%%%%%%%%%%%%%%%%%%%%%%%%%%%%%%%%%%%%%%%%%%%%%%%%%%%%%
\newcommand\DrawBitBoxUnsignedPicture[1]{
	\BeginTikzPicture
	\DrawBitBoxUnsigned{#1}
	\EndTikzPicture
}

%%%%%%%%%%%%%%%%%%%%%%%%%%%%%%%%%%%%%%%%%%%%%%%%%%%%%%%%%%%%%
\newcommand\DrawBitBoxSignedPicture[1]{
    \BeginTikzPicture
    \DrawBitBoxUnsigned{#1}
	% draw a box around the sign bit
	\draw {[rounded corners=\SignBoxCornerRadius] (1.35, -.6) -- (1.35, .6) -- (.65, .6) -- (.65, -.6) -- cycle};
    \EndTikzPicture
}


%%%%%%%%%%%%%%%%%%%%%%%%%%%%%%%%%%%%%%%%%%%%%%%%%%%%%%%%%%%%%
% #1 The total (extended) size
% #2 The value to use for left-side padding
% #3 The string to extend
\newcommand\DrawBitBoxLeftExtended[3]{
	\StrLen{#3}[\numchars]
	\pgfmathsetmacro\fill{int(#1-\numchars)}
	\begin{scope}[shift={(\fill,3.5)}]
    \DrawBitBoxUnsigned{#3}

   	% XXX IFF not zero-extending then draw a box around the sign bit
   	\draw {[rounded corners=\SignBoxCornerRadius] (1.35, -.6) -- (1.35, .6) -- (.65, .6) -- (.65, -.6) -- cycle};
	\end{scope}

	\DrawBitBox{#1}
	\DrawDoubleExtendedBitstring{#1}{0}{#3}{#2}{x}
	
   	% XXX IFF not zero-extending then draw a box around the sign bit
   	\draw {[rounded corners=\SignBoxCornerRadius] (\fill+1.35, -.6) -- (\fill+1.35, .6) -- (\fill+.65, .6) -- (\fill+.65, -.6) -- cycle};
    % draw a box around the extended sign bits
    \draw (.65, -.6) -- (.65, .6) -- (\fill+.35, .6) -- (\fill+.35, -.6) -- cycle;


}

%%%%%%%%%%%%%%%%%%%%%%%%%%%%%%%%%%%%%%%%%%%%%%%%%%%%%%%%%%%%%
\newcommand\DrawBitBoxSignExtendedPicture[2]{
	\BeginTikzPicture
	\DrawBitBoxLeftExtended{#1}{\substring{#2}{1}{1}}{#2}
    \EndTikzPicture
}

%%%%%%%%%%%%%%%%%%%%%%%%%%%%%%%%%%%%%%%%%%%%%%%%%%%%%%%%%%%%%
\newcommand\DrawBitBoxZeroExtendedPicture[2]{
	\BeginTikzPicture
	\DrawBitBoxLeftExtended{#1}{0}{#2}
    \EndTikzPicture
}


%%%%%%%%%%%%%%%%%%%%%%%%%%%%%%%%%%%%%%%%%%%%%%%%%%%%%%%%%%%%%
% #1 Total bit length
% #2 The string to print
% #3 Right-side padding length
\newcommand\DrawBitBoxSignLeftZeroRightExtendedPicture[3]{
	\BeginTikzPicture

	\StrLen{#2}[\numchars]
	\pgfmathsetmacro\fill{int(#1-\numchars-#3)}
	\begin{scope}[shift={(\fill,3.5)}]
    \DrawBitBoxUnsigned{#2}
    % draw a box around the sign bit
    %\draw (1.35, -.6) -- (1.35, .6) -- (.65, .6) -- (.65, -.6) -- cycle;
    \draw {[rounded corners=\SignBoxCornerRadius] (1.35, -.6) -- (1.35, .6) -- (.65, .6) -- (.65, -.6) -- cycle};
	\end{scope}

	\DrawBitBox{#1}
	\DrawDoubleExtendedBitstring{#1}{#3}{#2}{\substring{#2}{1}{1}}{0}

	% Box the sign bit
    \draw {[rounded corners=\SignBoxCornerRadius] (\fill+1.35, -.6) -- (\fill+1.35, .6) -- (\fill+.65, .6) -- (\fill+.65, -.6) -- cycle};

	\ifthenelse{\fill > 0}
	{
    	% Box the left-extended sign bits
    	\draw (.65, -.6) -- (.65, .6) -- (\fill+.35, .6) -- (\fill+.35, -.6) -- cycle;
		% \fill[blue!40!white] (.65, -.6) rectangle (\fill-.25, 1.2);
	}
	{}
	\ifthenelse{#3 > 0}
	{
    	% Box the right-extended sign bits
		\pgfmathsetmacro\posn{int(\numchars+\fill)}
    	\draw (\posn+.65, -.6) -- (\posn+.65, .6) -- (\posn+#3+.35, .6) -- (\posn+#3+.35, -.6) -- cycle;
	}
	{}
	

    \EndTikzPicture
}


%%%%%%%%%%%%%%%%%%%%%%%%%%%%%%%%%%%%%%%%%%%%%%%%%%%%%%%%%%%%%
%%%%%%%%%%%%%%%%%%%%%%%%%%%%%%%%%%%%%%%%%%%%%%%%%%%%%%%%%%%%%
%%%%%%%%%%%%%%%%%%%%%%%%%%%%%%%%%%%%%%%%%%%%%%%%%%%%%%%%%%%%%
%
% Draw hex markers
% #1 The number of bits in the box
\newcommand\DrawHexMarkers[1]{
	\pgfmathsetmacro\num{int(#1-1)}
	\foreach \x in {4,8,...,\num}
		\draw [line width=.5mm] (\x+.5,-.75) -- (\x+.5, -.3);
}


% Print the characters within a string evenly spaced at integral node positions
%
% #1 The number of characters in the string
% #2 The string of characters to plot
\newcommand\DrawInsnBitstring[3]{
	\pgfmathsetmacro\num{int(#1-1)}
	\foreach \x in {1,2,...,#1}
    	\draw(\x,0) node{\substring{#2}{\x}{\x}};

	\draw(#1+1,0) node[right]{#3};
}

%%%%%%%%%%%%%%%%%%%%%%%%%%%%%%%%%%%%%%%%%%%%%%%%%%%%%%%%%%%%%
% Draw a bit-separator line with labels at the given bit-offset (from the right)
%
% #1 Total box width
% #2 The position that the sepatator will be drawn to the left.
\newcommand\DrawInsnBoxSep[2]{
	\draw (#1-#2-.5,-.75) -- (#1-#2-.5, 1.5);
	\node at (#1-#2,1.2) {\tiny#2};
	\pgfmathsetmacro\result{int(#2+1)}
	\node at (#1-#2-1,1.2) {\tiny\result};
}

%%%%%%%%%%%%%%%%%%%%%%%%%%%%%%%%%%%%%%%%%%%%%%%%%%%%%%%%%%%%%
% #1 total characters/width
% #2 MSB position
% #3 LSB position
% #4 the segment label
\newcommand\DrawInsnBoxSeg[4]{
	\pgfmathsetmacro\leftpos{int(#1-#2)}
	\pgfmathsetmacro\rightpos{int(#1-#3)}

	\draw (\leftpos-.5,-.75) -- (\rightpos+.5,-.75);	% box bottom
	\draw (\leftpos-.5,1.75) -- (\rightpos+.5,1.75);	% box top
	\draw (\leftpos-.5,-.75) -- (\leftpos-.5, 2.5);		% left end
	\draw (\rightpos+.5,-.75) -- (\rightpos+.5, 2.5);	% right end
	\node at (\leftpos,2.2) {\tiny#2};
	\draw(\rightpos,2.2) node{\tiny#3};

	\pgfmathsetmacro\posn{#1-#2+(#2-#3)/2}
	\pgfmathsetmacro\range{int(#2-#3+1)}
	\node at (\posn,-1.4) {\small\range};		% the field width
	\node at (\posn,1.2) {\small#4};			% the field label

%	% arrows showing the span of the bits... meh
%    \draw[->] (\posn+.5,-1.4) -- (\rightpos+.2,-1.4);
%    \draw[->] (\posn-.5,-1.4) -- (\leftpos-.2,-1.4);
}

%%%%%%%%%%%%%%%%%%%%%%%%%%%%%%%%%%%%%%%%%%%%%%%%%%%%%%%%%%%%%
%%%%%%%%%%%%%%%%%%%%%%%%%%%%%%%%%%%%%%%%%%%%%%%%%%%%%%%%%%%%%
\newcommand\InsnStatement[1]{
%	\textbf{\large #1}\\
%	\textbf{#1}\\
	{\large #1}
}
%%%%%%%%%%%%%%%%%%%%%%%%%%%%%%%%%%%%%%%%%%%%%%%%%%%%%%%%%%%%%
% #1 the binary encoding
\newcommand\DrawInsnTypeBTikz[1]{
	\BeginTikzPicture
	\StrLen{#1}[\numchars]
	\DrawInsnBitstring{\numchars}{#1}{\hyperref[insnformat:btype]{B-type}}
	\DrawInsnBoxSeg{\numchars}{31}{25}{imm[12\textbar10:5]}
	\DrawInsnBoxSeg{\numchars}{24}{20}{rs2}
	\DrawInsnBoxSeg{\numchars}{19}{15}{rs1}
	\DrawInsnBoxSeg{\numchars}{14}{12}{funct3}
	\DrawInsnBoxSeg{\numchars}{11}{7}{imm[4:1\textbar11]}
	\DrawInsnBoxSeg{\numchars}{6}{0}{opcode}

	% add some hint bits in for imm fields
	\draw {[rounded corners=\SignBoxCornerRadius] (1.35, -.6) -- (1.35, .6) -- (.65, .6) -- (.65, -.6) -- cycle};	% sign bit
	\draw (32-7-.5, -.75) -- (32-7-.5, .1);		% imm[11]

	\DrawHexMarkers{\numchars}
	\EndTikzPicture
}

\newcommand\DrawInsnTypeBPicture[2]{
	\InsnStatement{#1}\\
	\DrawInsnTypeBTikz{#2}
}


%%%%%%%%%%%%%%%%%%%%%%%%%%%%%%%%%%%%%%%%%%%%%%%%%%%%%%%%%%%%%
% #1 the binary encoding
\newcommand\DrawInsnTypeUTikz[1]{
	\BeginTikzPicture
	\StrLen{#1}[\numchars]
	\DrawInsnBitstring{\numchars}{#1}{\hyperref[insnformat:utype]{U-type}}
	\DrawInsnBoxSeg{\numchars}{31}{12}{imm[31:12]}
	\DrawInsnBoxSeg{\numchars}{11}{7}{rd}
	\DrawInsnBoxSeg{\numchars}{6}{0}{opcode}

	% add some hint bits in for imm fields
	\draw {[rounded corners=\SignBoxCornerRadius] (1.35, -.6) -- (1.35, .6) -- (.65, .6) -- (.65, -.6) -- cycle};	% sign bit

	\DrawHexMarkers{\numchars}
	\EndTikzPicture
}

\newcommand\DrawInsnTypeUPicture[2]{
	\InsnStatement{#1}\\
	\DrawInsnTypeUTikz{#2}
}


%%%%%%%%%%%%%%%%%%%%%%%%%%%%%%%%%%%%%%%%%%%%%%%%%%%%%%%%%%%%%
% #1 the binary encoding
\newcommand\DrawInsnTypeJTikz[1]{
	\BeginTikzPicture
	\StrLen{#1}[\numchars]
	\DrawInsnBitstring{\numchars}{#1}{\hyperref[insnformat:jtype]{J-type}}
	\DrawInsnBoxSeg{\numchars}{31}{12}{imm[20\textbar10:1\textbar11\textbar19:12]}
	\DrawInsnBoxSeg{\numchars}{11}{7}{rd}
	\DrawInsnBoxSeg{\numchars}{6}{0}{opcode}

	% add some hint bits in for imm fields
	\draw {[rounded corners=\SignBoxCornerRadius] (1.35, -.6) -- (1.35, .6) -- (.65, .6) -- (.65, -.6) -- cycle};	% sign bit
	\draw (32-19-.5, -.75) -- (32-19.5, .1);		% imm[19:12]
	\draw (32-20-.5, -.75) -- (32-20.5, .1);		% imm[11]

	\DrawHexMarkers{\numchars}
	\EndTikzPicture
}

\newcommand\DrawInsnTypeJPicture[2]{
	\InsnStatement{#1}\\
	\DrawInsnTypeJTikz{#2}
}

%%%%%%%%%%%%%%%%%%%%%%%%%%%%%%%%%%%%%%%%%%%%%%%%%%%%%%%%%%%%%
% #1 the binary encoding
\newcommand\DrawInsnTypeI[1]{
	\StrLen{#1}[\numchars]
	\DrawInsnBitstring{\numchars}{#1}{\hyperref[insnformat:itype]{I-type}}
	\DrawInsnBoxSeg{\numchars}{31}{20}{imm[11:0]}
	\DrawInsnBoxSeg{\numchars}{19}{15}{rs1}
	\DrawInsnBoxSeg{\numchars}{14}{12}{funct3}
	\DrawInsnBoxSeg{\numchars}{11}{7}{rd}
	\DrawInsnBoxSeg{\numchars}{6}{0}{opcode}

	% add some hint bits in for imm fields
	\draw {[rounded corners=\SignBoxCornerRadius] (1.35, -.6) -- (1.35, .6) -- (.65, .6) -- (.65, -.6) -- cycle};	% sign bit
}
%%%%%%%%%%%%%%%%%%%%%%%%%%%%%%%%%%%%%%%%%%%%%%%%%%%%%%%%%%%%%
% #1 the binary encoding
\newcommand\DrawInsnTypeITikz[1]{
	\BeginTikzPicture
	\DrawInsnTypeI{#1}
	\DrawHexMarkers{\numchars}
	\EndTikzPicture
}
\newcommand\DrawInsnTypeIPicture[2]{
	\InsnStatement{#1}\\
	\DrawInsnTypeITikz{#2}
}

%%%%%%%%%%%%%%%%%%%%%%%%%%%%%%%%%%%%%%%%%%%%%%%%%%%%%%%%%%%%%
% #1 the binary encoding
\newcommand\DrawInsnTypeSTikz[1]{
	\BeginTikzPicture
	\StrLen{#1}[\numchars]
	\DrawInsnBitstring{\numchars}{#1}{\hyperref[insnformat:stype]{S-type}}
	\DrawInsnBoxSeg{\numchars}{31}{25}{imm[11:5]}
	\DrawInsnBoxSeg{\numchars}{24}{20}{rs2}
	\DrawInsnBoxSeg{\numchars}{19}{15}{rs1}
	\DrawInsnBoxSeg{\numchars}{14}{12}{funct3}
	\DrawInsnBoxSeg{\numchars}{11}{7}{imm[4:0]}
	\DrawInsnBoxSeg{\numchars}{6}{0}{opcode}

	% add some hint bits in for imm fields
	\draw {[rounded corners=\SignBoxCornerRadius] (1.35, -.6) -- (1.35, .6) -- (.65, .6) -- (.65, -.6) -- cycle};	% sign bit

	\DrawHexMarkers{\numchars}
	\EndTikzPicture
}

\newcommand\DrawInsnTypeSPicture[2]{
	\InsnStatement{#1}\\
	\DrawInsnTypeSTikz{#2}
}

%%%%%%%%%%%%%%%%%%%%%%%%%%%%%%%%%%%%%%%%%%%%%%%%%%%%%%%%%%%%%
% #1 the binary encoding
\newcommand\DrawInsnTypeRShiftTikz[1]{
	\BeginTikzPicture
	\StrLen{#1}[\numchars]
	\DrawInsnBitstring{\numchars}{#1}{\hyperref[insnformat:rtype]{R-type}}
	\DrawInsnBoxSeg{\numchars}{31}{25}{funct7}
	\DrawInsnBoxSeg{\numchars}{24}{20}{shamt}
	\DrawInsnBoxSeg{\numchars}{19}{15}{rs1}
	\DrawInsnBoxSeg{\numchars}{14}{12}{funct3}
	\DrawInsnBoxSeg{\numchars}{11}{7}{rd}
	\DrawInsnBoxSeg{\numchars}{6}{0}{opcode}

	\DrawHexMarkers{\numchars}
	\EndTikzPicture
}

\newcommand\DrawInsnTypeRShiftPicture[2]{
	\InsnStatement{#1}\\
	\DrawInsnTypeRShiftTikz{#2}
}

%%%%%%%%%%%%%%%%%%%%%%%%%%%%%%%%%%%%%%%%%%%%%%%%%%%%%%%%%%%%%
% #1 the binary encoding
\newcommand\DrawInsnTypeRTikz[1]{
	\BeginTikzPicture
	\StrLen{#1}[\numchars]
	\DrawInsnBitstring{\numchars}{#1}{\hyperref[insnformat:rtype]{R-type}}
	\DrawInsnBoxSeg{\numchars}{31}{25}{funct7}
	\DrawInsnBoxSeg{\numchars}{24}{20}{rs2}
	\DrawInsnBoxSeg{\numchars}{19}{15}{rs1}
	\DrawInsnBoxSeg{\numchars}{14}{12}{funct3}
	\DrawInsnBoxSeg{\numchars}{11}{7}{rd}
	\DrawInsnBoxSeg{\numchars}{6}{0}{opcode}

	\DrawHexMarkers{\numchars}
	\EndTikzPicture
}

\newcommand\DrawInsnTypeRPicture[2]{
	\InsnStatement{#1}\\
	\DrawInsnTypeRTikz{#2}
}

%%%%%%%%%%%%%%%%%%%%%%%%%%%%%%%%%%%%%%%%%%%%%%%%%%%%%%%%%%%%%
% #1 the binary encoding
\newcommand\DrawInsnTypeFTikz[1]{
    \BeginTikzPicture
    \StrLen{#1}[\numchars]
    \DrawInsnBitstring{\numchars}{#1}{FENCE}
    \DrawInsnBoxSeg{\numchars}{31}{28}{}
    \DrawInsnBoxSeg{\numchars}{27}{24}{pred}
    \DrawInsnBoxSeg{\numchars}{23}{20}{succ}
    \DrawInsnBoxSeg{\numchars}{19}{15}{}
    \DrawInsnBoxSeg{\numchars}{14}{12}{funct3}
    \DrawInsnBoxSeg{\numchars}{11}{7}{}
    \DrawInsnBoxSeg{\numchars}{6}{0}{opcode}

    \DrawHexMarkers{\numchars}
    \EndTikzPicture
}

\newcommand\DrawInsnTypeFPicture[2]{
    \InsnStatement{#1}\\
    \DrawInsnTypeFTikz{#2}
}

%%%%%%%%%%%%%%%%%%%%%%%%%%%%%%%%%%%%%%%%%%%%%%%%%%%%%%%%%%%%%
% #1 the binary encoding
\newcommand\DrawInsnTypeETikz[1]{
    \BeginTikzPicture
	\StrLen{#1}[\numchars]
	\DrawInsnBitstring{\numchars}{#1}{\hyperref[insnformat:itype]{I-type}}
	\DrawInsnBoxSeg{\numchars}{31}{20}{}
	\DrawInsnBoxSeg{\numchars}{19}{15}{}
	\DrawInsnBoxSeg{\numchars}{14}{12}{funct3}
	\DrawInsnBoxSeg{\numchars}{11}{7}{}
	\DrawInsnBoxSeg{\numchars}{6}{0}{opcode}

    \DrawHexMarkers{\numchars}
    \EndTikzPicture
}

\newcommand\DrawInsnTypeEPicture[2]{
    \InsnStatement{#1}\\
    \DrawInsnTypeETikz{#2}
}

%%%%%%%%%%%%%%%%%%%%%%%%%%%%%%%%%%%%%%%%%%%%%%%%%%%%%%%%%%%%%
% #1 the binary encoding
\newcommand\DrawInsnTypeCSTikz[1]{
    \BeginTikzPicture
	\StrLen{#1}[\numchars]
	\DrawInsnBitstring{\numchars}{#1}{\hyperref[insnformat:itype]{I-type}}
	\DrawInsnBoxSeg{\numchars}{31}{20}{csr}
	\DrawInsnBoxSeg{\numchars}{19}{15}{rs1}
	\DrawInsnBoxSeg{\numchars}{14}{12}{funct3}
	\DrawInsnBoxSeg{\numchars}{11}{7}{rd}
	\DrawInsnBoxSeg{\numchars}{6}{0}{opcode}

    \DrawHexMarkers{\numchars}
    \EndTikzPicture
}

\newcommand\DrawInsnTypeCSPicture[2]{
    \InsnStatement{#1}\\
    \DrawInsnTypeCSTikz{#2}
}

%%%%%%%%%%%%%%%%%%%%%%%%%%%%%%%%%%%%%%%%%%%%%%%%%%%%%%%%%%%%%
% #1 the binary encoding
\newcommand\DrawInsnTypeCSITikz[1]{
    \BeginTikzPicture
	\StrLen{#1}[\numchars]
	\DrawInsnBitstring{\numchars}{#1}{\hyperref[insnformat:itype]{I-type}}
	\DrawInsnBoxSeg{\numchars}{31}{20}{csr}
	\DrawInsnBoxSeg{\numchars}{19}{15}{zimm}
	\DrawInsnBoxSeg{\numchars}{14}{12}{funct3}
	\DrawInsnBoxSeg{\numchars}{11}{7}{rd}
	\DrawInsnBoxSeg{\numchars}{6}{0}{opcode}

    \DrawHexMarkers{\numchars}
    \EndTikzPicture
}

\newcommand\DrawInsnTypeCSIPicture[2]{
    \InsnStatement{#1}\\
    \DrawInsnTypeCSITikz{#2}
}



%%%%%%%%%%%%%%%%%%%%%%%%%%%%%%%%%%%%%%%%%%%%%%%%%%%%%%%%%%%%%%%%%%%%%%%
%%%%%%%%%%%%%%%%%%%%%%%%%%%%%%%%%%%%%%%%%%%%%%%%%%%%%%%%%%%%%%%%%%%%%%%
%%%%%%%%%%%%%%%%%%%%%%%%%%%%%%%%%%%%%%%%%%%%%%%%%%%%%%%%%%%%%%%%%%%%%%%
%%%%%%%%%%%%%%%%%%%%%%%%%%%%%%%%%%%%%%%%%%%%%%%%%%%%%%%%%%%%%%%%%%%%%%%


\newcommand\xTInsnStatement[4]{%
	\parbox{3.5cm}{{\sffamily\large\bfseries #2}\\
	\tt#3}\hspace{5mm}\parbox{5cm}{\bfseries#1}\parbox{12cm}{#4}%
}

\newcommand\TInsnStatement[4]{%
	\begin{tabular}{lll}
	\parbox[t]{3.5cm}{{\sffamily\large\bfseries #2}\\
	\tt#3} & \parbox[t]{5cm}{\bfseries #1} & \parbox[t]{12cm}{#4}\\
	\end{tabular}
}

\newcommand\TDrawInsnTypeUPicture[5]{%
	\TInsnStatement{#1}{#2}{#3}{#4}\\
    \DrawInsnTypeUTikz{#5}%
}

\newcommand\TDrawInsnTypeJPicture[5]{%
	\TInsnStatement{#1}{#2}{#3}{#4}\\
    \DrawInsnTypeJTikz{#5}%
}

\newcommand\TDrawInsnTypeBPicture[5]{%
	\TInsnStatement{#1}{#2}{#3}{#4}\\
    \DrawInsnTypeBTikz{#5}%
}

\newcommand\TDrawInsnTypeIPicture[5]{%
	\TInsnStatement{#1}{#2}{#3}{#4}\\
    \DrawInsnTypeITikz{#5}%
}
\newcommand\TDrawInsnTypeSPicture[5]{%
	\TInsnStatement{#1}{#2}{#3}{#4}\\
    \DrawInsnTypeSTikz{#5}%
}
\newcommand\TDrawInsnTypeRPicture[5]{%
	\TInsnStatement{#1}{#2}{#3}{#4}\\
    \DrawInsnTypeRTikz{#5}%
}

\newcommand\TDrawInsnTypeRShiftPicture[5]{
	\TInsnStatement{#1}{#2}{#3}{#4}\\
	\DrawInsnTypeRShiftTikz{#5}
}



%%%%%%%%%%%%%%%%%%%%%%%%%%%%%%%%%%%%%%%%%%%%%%%%%%%%%%%%%%%%%%%%%%%%%%%
%%%%%%%%%%%%%%%%%%%%%%%%%%%%%%%%%%%%%%%%%%%%%%%%%%%%%%%%%%%%%%%%%%%%%%%
%%%%%%%%%%%%%%%%%%%%%%%%%%%%%%%%%%%%%%%%%%%%%%%%%%%%%%%%%%%%%%%%%%%%%%%
%%%%%%%%%%%%%%%%%%%%%%%%%%%%%%%%%%%%%%%%%%%%%%%%%%%%%%%%%%%%%%%%%%%%%%%


% print a register name in typewriter font
\newcommand\reg[1]{{\tt #1}}

% print an instruction name in typewriter font
\newcommand\insn[1]{{\tt #1}}

\newcommand\rvddt{{\tt rvddt}}

\newcommand\hex[1]{{\tt 0x#1}}
