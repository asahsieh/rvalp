\chapter{Installing a RISC-V Toolchain}
\label{chapter:install}

All of the software presented in this text was assembled/compiled 
using the GNU toolchain and executed using the rvddt simulator on 
a Linux (Ubuntu 20.04 LTS) operating system.

The installation instructions provided here were last tested on 
on March 5, 2021.

It is expected that these tools will evolve over time.  See the
respective documentation web sites for the latest news and options
for installing them.

%%%%%%%%%%%%%%%%%%%%%%%%%%%%%%%%%%%%%%%%%%%%%%%%%%%%%%%%%%%%%%%%%%%%
%%%%%%%%%%%%%%%%%%%%%%%%%%%%%%%%%%%%%%%%%%%%%%%%%%%%%%%%%%%%%%%%%%%%
%%%%%%%%%%%%%%%%%%%%%%%%%%%%%%%%%%%%%%%%%%%%%%%%%%%%%%%%%%%%%%%%%%%%
\section{The GNU Toolchain}

\enote{It would be good to find some Mac and Windows users to write 
and test proper variations on this section to address those systems.
Pull requests, welcome!}%
In order to install custom code in a location that will not cause 
interference with other applications (and allow for easy hacking and
cleanup), these will install the toolchain under
a private directory: \verb@~/projects/riscv/install@.  At any time
you can remove everything and start over by executing the following 
command:

\begin{tty}
rm -rf ~/projects/riscv/install
\end{tty}

\begin{tcolorbox}
Be {\em very} careful how you type the above \verb@rm@ command.  
If typed incorrectly, it could irreversibly remove many of your files!
\end{tcolorbox}

Before building the toolchain, a number of utilities must be present on 
your system.  The following will install those that are needed:

\begin{minipage}{\textwidth}
\begin{tty}
sudo apt install autoconf automake autotools-dev curl python3 python-dev libmpc-dev \
    libmpfr-dev libgmp-dev gawk build-essential bison flex texinfo gperf \
    libtool patchutils bc zlib1g-dev libexpat-dev
\end{tty}
\end{minipage}

Note that the above \verb@apt@ command is the only operation that should
be performed as root.  All other commands should be executed as a regular 
user.  This will eliminate the possibility of clobbering system files that 
should not be touched when tinkering with the toolchain applications.

\enote{Discuss the choice of ilp32 as well as what the other variations 
would do.}%
To download, compile and install the toolchain:

\begin{minipage}{\textwidth}
\begin{tty}
mkdir -p ~/projects/riscv
cd ~/projects/riscv
git clone https://github.com/riscv/riscv-gnu-toolchain
cd riscv-gnu-toolchain
INS_DIR=~/projects/riscv/install/rv32i
./configure --prefix=$INS_DIR \
    --with-multilib-generator="rv32i-ilp32--;rv32imafd-ilp32--;rv32ima-ilp32--"
make
\end{tty}
\end{minipage}

After building the toolchain, make it available by putting it into
your PATH by adding the following to the end of your \verb@.bashrc@ file:

\begin{tty}
export PATH=$PATH:$INS_DIR
\end{tty}

For this \verb@PATH@ change to take place, start a new terminal or paste the
same \verb@export@ command into your existing terminal.



%%%%%%%%%%%%%%%%%%%%%%%%%%%%%%%%%%%%%%%%%%%%%%%%%%%%%%%%%%%%%%%%%%%%
%%%%%%%%%%%%%%%%%%%%%%%%%%%%%%%%%%%%%%%%%%%%%%%%%%%%%%%%%%%%%%%%%%%%
%%%%%%%%%%%%%%%%%%%%%%%%%%%%%%%%%%%%%%%%%%%%%%%%%%%%%%%%%%%%%%%%%%%%
\section{rvddt}

Download and install the rvddt simulator by executing the following 
commands.
Building the rvddt example programs will verify that the GNU toolchain
has been built and installed properly.

\begin{minipage}{\textwidth}
%git clone git@github.com:johnwinans/rvddt.git
\begin{tty}
cd ~/projects/riscv
git clone https://github.com/johnwinans/rvddt.git
cd rvddt/src
make world
cd ../examples
make world
\end{tty}
\end{minipage}

After building rvddt, make it available by putting it into your PATH 
by adding the following to the end of your \verb@.bashrc@ file:

\begin{tty}
export PATH=$PATH:~/projects/riscv/rvddt/src
\end{tty}

For this \verb@PATH@ change to take place, start a new terminal or paste the
same \verb@export@ command into your existing terminal.


Test the rvddt build by executing one of the examples:

\begin{minipage}{\textwidth}
\begin{tty}
winans@ux410:~/projects/riscv/rvddt/examples$ rvddt -f counter/counter.bin
sp initialized to top of memory: 0x0000fff0
Loading 'counter/counter.bin' to 0x0
This is rvddt.  Enter ? for help.
ddt> ti 0 1000
00000000: 00300293  addi    x5, x0, 3     # x5 = 0x00000003 = 0x00000000 + 0x00000003
00000004: 00000313  addi    x6, x0, 0     # x6 = 0x00000000 = 0x00000000 + 0x00000000
00000008: 00130313  addi    x6, x6, 1     # x6 = 0x00000001 = 0x00000000 + 0x00000001
0000000c: fe534ee3  blt     x6, x5, -4    # pc = (0x1 < 0x3) ? 0x8 : 0x10
00000008: 00130313  addi    x6, x6, 1     # x6 = 0x00000002 = 0x00000001 + 0x00000001
0000000c: fe534ee3  blt     x6, x5, -4    # pc = (0x2 < 0x3) ? 0x8 : 0x10
00000008: 00130313  addi    x6, x6, 1     # x6 = 0x00000003 = 0x00000002 + 0x00000001
0000000c: fe534ee3  blt     x6, x5, -4    # pc = (0x3 < 0x3) ? 0x8 : 0x10
00000010: ebreak
ddt> x
winans@ux410:~/projects/riscv/rvddt/examples$ 
\end{tty}
\end{minipage}


%%%%%%%%%%%%%%%%%%%%%%%%%%%%%%%%%%%%%%%%%%%%%%%%%%%%%%%%%%%%%%%%%%%%
%%%%%%%%%%%%%%%%%%%%%%%%%%%%%%%%%%%%%%%%%%%%%%%%%%%%%%%%%%%%%%%%%%%%
%%%%%%%%%%%%%%%%%%%%%%%%%%%%%%%%%%%%%%%%%%%%%%%%%%%%%%%%%%%%%%%%%%%%
\section{qemu}

You can download and install the RV32 qemu simulator by executing 
the following commands.  

At the time of this writing (2021-06) I use release v5.0.0.  
Release v5.2.0 has issues that confuse GDB when printing the registers 
and v6.0.0 has different CPU types that I have had trouble with when 
executing privileged instructions.

\begin{minipage}{\textwidth}
\begin{tty}
INS_DIR=~/projects/riscv/install/rv32i
cd ~/projects/riscv
git clone git@github.com:qemu/qemu.git
cd qemu
git checkout v5.0.0
./configure --target-list=riscv32-softmmu --prefix=${INS_DIR}
make -j4
make install
\end{tty}
\end{minipage}

