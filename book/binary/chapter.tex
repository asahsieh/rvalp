\chapter{Numbers and Storage Systems}
\label{chapter:numbers}

This chapter discusses how data are represented and stored in a computer.

In the context of computing, {\em boolean} refers to a condition that can 
be either true and false and {\em binary} refers to the use of a base-2 
numeric system to rpresent numbers.

RISC-V assembly language uses binary to represent all values, be they 
boolean or numeric.  It is the context within which they are used that
determines whether they are boolean or numeric.

\enote{Add some diagrams here showing bits, bytes and the MSB, 
LSB,\ldots\ perhaps relocated from the RV32I chapter?}

%%%%%%%%%%%%%%%%%%%%%%%%%%%%%%%%%%%%%%%%%%%%%%%%%%%%%%%%%%%%%%%%%%%%%%%%%
%%%%%%%%%%%%%%%%%%%%%%%%%%%%%%%%%%%%%%%%%%%%%%%%%%%%%%%%%%%%%%%%%%%%%%%%%
\section{Boolean Functions}

Boolean functions apply on a per-bit basis.
When applied to multi-bit values, each bit position is operated upon 
independently of the other bits.

RISC-V assembly language uses zero to represent {\em false} and one 
to represent {\em true}.  In general, however, it is useful to relax 
this and define zero {\bf and only zero} to be {\em false} and anything 
that is not {\em false} is therefore {\em true}.%
\footnote{This is how {\em true} and {\em false} behave in C, C++, and
many other languages as well as the common assembly language idioms
discussed in this text.}

The reason for this relaxation is because, while a single binary digit 
(\gls{bit}) can represent the two values zero and one, the vast majority 
of the time data is processed by the CPU in groups of bits.  These
groups have names like \gls{byte} (8 bits), \gls{halfword} (16 bits) 
and \gls{fullword} (32 bits).

%%%%%%%%%%%%%%%%%%%%%%%%%%%%%%%%%%%%%%%%%%%%%%%%%%%%%%%%%%%%%%%%%%%%%%%%%
%%%%%%%%%%%%%%%%%%%%%%%%%%%%%%%%%%%%%%%%%%%%%%%%%%%%%%%%%%%%%%%%%%%%%%%%%%
\subsection{NOT}

The {\em NOT} operator applies to a single operand and represents the 
opposite of the input.
\enote{Need to define unary, binary and ternary operators without 
confusing binary operators with binary numbers.}

If the input is 1 then the output is 0.  If the input is 0 then the
output is 1.  In other words, the output value is {\em not} that of the 
input value.

Expressing the {\em not} function in the form a a truth table:

\begin{center}
\begin{tabular}{c|c}
A & $\overline{\mbox{A}}$\\
\hline
0 & 1 \\
1 & 0 \\
\end{tabular}
\end{center}

A truth table is drawn by indicating all of the possible input values on 
the left of the vertical bar with each row displaying the output values 
that correspond to the input for that row.  The column headings are used
to define the illustrated operation expressed using a mathematical 
notation.  The {\em not} operation is indicated by the presense of
an {\em overline}.

In computer programming languages, things like an overline can not be 
efficiently expressed using a standard keyboard.  Therefore it is common 
to use a notation such as that used  by the C language when discussing 
the {\em NOT} operator in symbolic form.  Specifically the tilde: `\verb@~@'.

It is also uncommon to for programming languages to express boolean operations 
on single-bit input(s).  A more generalized operation is used that applies 
to a set of bits all at once.  For example, performing a {\em not} operation
of eight bits at once can be illustrated as:


\begin{verbatim}
  ~ 1 1 1 1 0 1 0 1  <== A
  -----------------
    0 0 0 0 1 0 1 0  <== output
\end{verbatim}

In a line of code the above might read like this: \verb@output = ~A@

%%%%%%%%%%%%%%%%%%%%%%%%%%%%%%%%%%%%%%%%%%%%%%%%%%%%%%%%%%%%%%%%%%%%%%%%%
%%%%%%%%%%%%%%%%%%%%%%%%%%%%%%%%%%%%%%%%%%%%%%%%%%%%%%%%%%%%%%%%%%%%%%%%%%
\subsection{AND}

The boolean {\em and} function has two or more inputs and the output is a
single bit.  The output is 1 if and only if all of the input values are 1.
Otherwise it is 0.

This function works like it does in spoken language.  For example
if A is 1 {\em AND} B is 1 then the output is 1 (true).
Otherwise the output is 0 (false).  

In mathamatical notion, the {\em and} operator is expressed the same way
as is {\em multiplication}.  That is by a raised dot between, or by 
juxtaposition of, two variable names.  It is also worth noting that,
in base-2, the {\em and} operation actually {\em is} multiplication!

\begin{center}
\begin{tabular}{cc|c}
A & B & AB \\
\hline
0 & 0 & 0 \\
0 & 1 & 0 \\
1 & 0 & 0 \\
1 & 1 & 1 \\
\end{tabular}
\end{center}

This text will use the operator used in the C language when discussing 
the {\em AND} operator in symbolic form.  Specifically the ampersand: `\verb@&@'.

An eight-bit example:

\begin{verbatim}
    1 1 1 1 0 1 0 1  <== A
  & 1 0 0 1 0 0 1 1  <== B
  -----------------
    1 0 0 1 0 0 0 1  <== output
\end{verbatim}

In a line of code the above might read like this: \verb@output = A & B@

%%%%%%%%%%%%%%%%%%%%%%%%%%%%%%%%%%%%%%%%%%%%%%%%%%%%%%%%%%%%%%%%%%%%%%%%%
%%%%%%%%%%%%%%%%%%%%%%%%%%%%%%%%%%%%%%%%%%%%%%%%%%%%%%%%%%%%%%%%%%%%%%%%%%
\subsection{OR}

The boolean {\em or} function has two or more inputs and the output is a
single bit.  The output is 1 if at least one of the input values are 1.

This function works like it does in spoken language.  For example
if A is 1 {\em OR} B is 1 then the output is 1 (true).
Otherwise the output is 0 (false).  

In mathamatical notion, the {\em or} operator is expressed using the plus 
($+$).  

\begin{center}
\begin{tabular}{cc|c}
A & B & A$+$B \\
\hline
0 & 0 & 0 \\
0 & 1 & 1 \\
1 & 0 & 1 \\
1 & 1 & 1 \\
\end{tabular}
\end{center}

This text will use the operator used in the C language when discussing
the {\em OR}  operator in symbolic form.  Specifically the pipe: `\verb@|@'.

An eight-bit example:

\begin{verbatim}
    1 1 1 1 0 1 0 1  <== A
  | 1 0 0 1 0 0 1 1  <== B
  -----------------
    1 1 1 1 0 1 1 1  <== output
\end{verbatim}

In a line of code the above might read like this: \verb@output = A | B@

%%%%%%%%%%%%%%%%%%%%%%%%%%%%%%%%%%%%%%%%%%%%%%%%%%%%%%%%%%%%%%%%%%%%%%%%%
%%%%%%%%%%%%%%%%%%%%%%%%%%%%%%%%%%%%%%%%%%%%%%%%%%%%%%%%%%%%%%%%%%%%%%%%%%
\subsection{XOR}

The boolean {\em exclusive or} function has two or more inputs and the 
output is a single bit.  The output is 1 if only an odd number of inputs 
are 1. Otherwise the output will be 0.

Note that when {\em XOR} is used with two inputs, the output
is set to 1 (true) when the inputs have different values and 0 
(false) when the inputs both have the same value.

In mathamatical notion, the {\em xor} operator is expressed using the plus
in a circle ($\oplus$).

\begin{center}
\begin{tabular}{cc|c}
A & B & A$\oplus{}$B \\
\hline
0 & 0 & 0 \\
0 & 1 & 1 \\
1 & 0 & 1 \\
1 & 1 & 0 \\
\end{tabular}
\end{center}

This text will use the operator used in the C language when discussing
the {\em XOR}  operator in symbolic form.  Specifically the carrot: `\verb@^@'.


An eight-bit example:

\begin{verbatim}
    1 1 1 1 0 1 0 1  <== A
  ^ 1 0 0 1 0 0 1 1  <== B
  -----------------
    0 1 1 0 0 1 1 0  <== output
\end{verbatim}

In a line of code the above might read like this: \verb@output = A ^ B@


%%%%%%%%%%%%%%%%%%%%%%%%%%%%%%%%%%%%%%%%%%%%%%%%%%%%%%%%%%%%%%%%%%%%%%%%%
%%%%%%%%%%%%%%%%%%%%%%%%%%%%%%%%%%%%%%%%%%%%%%%%%%%%%%%%%%%%%%%%%%%%%%%%%
%%%%%%%%%%%%%%%%%%%%%%%%%%%%%%%%%%%%%%%%%%%%%%%%%%%%%%%%%%%%%%%%%%%%%%%%%
\section{Integers and Counting}

A binary integer is constructed with only 1s and 0s in the same
manner as decimal numbers are constructed with values from 0 to 9.

Counting in binary (base-2) uses the same basic rules as decimal (base-10).  
The difference comes in when we consider that there are ten decimal digits and 
only two binary digits.  Therefore, in base-10, we must carry when adding one to 
nine (because there is no digit representing a ten) and, in base-2, we must 
carry when adding one to one (because there is no digit representing a two.)

\autoref{Figure:integers} shows an abridged table of the decimal, binary and hexadecimal 
values ranging from $0_{10}$ to $129_{10}$.

\begin{figure}[t]
\begin{center}
\begin{tabular}{|c|c|c||c|c|c|c|c|c|c|c||c|c|}
\hline
\multicolumn{3}{|c||}{Decimal} & \multicolumn{8}{|c||}{Binary} & \multicolumn{2}{|c|}{Hex}\\
\hline
$10^2$	& $10^1$	& $10^0$	& $2^7$ & $2^6$ & $2^5$ & $2^4$ & $2^3$ & $2^2$ & $2^1$ & $2^0$		& $16^1$ & $16^0$	\\
\hline
100		&	10		& 1			& 128	& 64	& 32	& 16	& 8		& 4		& 2		& 1			& 16	& 1			\\
\hline \hline
0		&	0		&	0		& 0 & 0 & 0 & 0 & 0 & 0 & 0 & 0		& 0 & 0 \\
0		&	0		&	1		& 0 & 0 & 0 & 0 & 0 & 0 & 0 & 1		& 0 & 1 \\
0		&	0		&	2		& 0 & 0 & 0 & 0 & 0 & 0 & 1 & 0		& 0 & 2 \\
0		&	0		&	3		& 0 & 0 & 0 & 0 & 0 & 0 & 1 & 1		& 0 & 3 \\
0		&	0		&	4		& 0 & 0 & 0 & 0 & 0 & 1 & 0 & 0		& 0 & 4 \\
0		&	0		&	5		& 0 & 0 & 0 & 0 & 0 & 1 & 0 & 1		& 0 & 5 \\
0		&	0		&	6		& 0 & 0 & 0 & 0 & 0 & 1 & 1 & 0		& 0 & 6 \\
0		&	0		&	7		& 0 & 0 & 0 & 0 & 0 & 1 & 1 & 1		& 0 & 7 \\
0		&	0		&	8		& 0 & 0 & 0 & 0 & 1 & 0 & 0 & 0		& 0 & 8 \\
0		&	0		&	9		& 0 & 0 & 0 & 0 & 1 & 0 & 0 & 1		& 0 & 9 \\
0		&	1		&	0		& 0 & 0 & 0 & 0 & 1 & 0 & 1 & 0		& 0 & a \\
0		&	1		&	1		& 0 & 0 & 0 & 0 & 1 & 0 & 1 & 1		& 0 & b \\
0		&	1		&	2		& 0 & 0 & 0 & 0 & 1 & 1 & 0 & 0		& 0 & c \\
0		&	1		&	3		& 0 & 0 & 0 & 0 & 1 & 1 & 0 & 1		& 0 & d \\
0		&	1		&	4		& 0 & 0 & 0 & 0 & 1 & 1 & 1 & 0		& 0 & e \\
0		&	1		&	5		& 0 & 0 & 0 & 0 & 1 & 1 & 1 & 1		& 0 & f \\
0		&	1		&	6		& 0 & 0 & 0 & 1 & 0 & 0 & 0 & 0		& 1 & 0 \\
0		&	1		&	7		& 0 & 0 & 0 & 1 & 0 & 0 & 0 & 1		& 1 & 1 \\
\hline
\multicolumn{3}{|c||}{\ldots} & \multicolumn{8}{|c||}{\ldots} & \multicolumn{2}{|c|}{\ldots}\\
\hline
1       &   2       &   5		& 0 & 1 & 1 & 1 & 1 & 1 & 0 & 1		& 7 & d \\
1       &   2       &   6		& 0 & 1 & 1 & 1 & 1 & 1 & 1 & 0		& 7 & e \\
1       &   2       &   7		& 0 & 1 & 1 & 1 & 1 & 1 & 1 & 1		& 7 & f \\
1       &   2       &   8		& 1 & 0 & 0 & 0 & 0 & 0 & 0 & 0		& 8 & 0 \\
\hline
\end{tabular}
\end{center}
\captionof{figure}{Counting in decimal, binary and hexadecimal.}
\label{Figure:integers}
\end{figure}

One way to look at this table is on a per-row basis where each place 
value is represented by the base raised to the power of the place value
position (shown in the column headings.)  
%This is useful when converting arbitrary numeric values between bases.  
For example to interpret the decimal value on the fourth row: 

\begin{equation}
0 \times 10^2 + 0 \times 10^1 + 3 \times 10^0 = 3_{10}
\end{equation}

Interpreting the binary value on the fourth row by converting it to decimal:

\begin{equation}
0 \times 2^7 + 0 \times 2^6 +0 \times 2^5 +0 \times 2^4 +0 \times 2^3 +0 \times 2^2 + 1 \times 2^1 + 1 \times 2^0 = 3_{10}
\end{equation}

Interpreting the hexadecimal value on the fourth row by converting it to decimal:

\begin{equation}
0 \times 16^1 + 3 \times 16^0 = 3_{10}
\end{equation}

\index{Most significant bit}\index{MSB|see {Most significant bit}}%
\index{Least significant bit}\index{LSB|see {Least significant bit}}%
We refer to the place values with the largest exponenet (the one furthest to the 
left for any given base) as the {\em most significant} digit and the place value
with the lowest exponenet as the {\em least significant} digit.  For binary
numbers these are the \acrfull{msb} and \acrfull{lsb} respectively.%
\footnote{Changing the value of the MSB will have a more {\em significant}
impact on the numeric value than changing the value of the LSB.} 


Another way to look at this table is on a per-column basis.  When
tasked with drawing such a table by hand, it might be useful 
to observe that, just as in decimal, the right-most column will
cycle through all of the values represented in the chosen base
then cycle back to zero and repeat.  (For example, in binary this
pattern is 0-1-0-1-0-1-0-\ldots) The next column in each base
will cycle in the same manner except each of the values is repeated
as many times as is represented by the place value (in the case of 
decimal, $10^1$ times, binary $2^1$ times, hex $16^1$ times.  Again,
the for binary numbers this pattern is  0-0-1-1-0-0-1-1-\ldots)
This continues for as many columns as are needed to represent the 
magnitude of the desired number.

Another item worth noting is that any even binary number will always
have a 0 LSB and odd numbers will always have a 1 LSB.

As is customary in decimal, leading zeros are sometimes not shown 
for readability.

The relationship between binary and hex values is also worth taking
note.  Because $2^4 = 16$, there is a clean and simple grouping
of 4 \gls{bit}s to 1 \gls{hit} (aka \gls{nybble}).  
There is no such relationship between binary and decimal.  

Writing and reading numbers in binary that are longer than 8 bits 
is cumbersome and prone to error.  The simple conversion between 
binary and hex makes hex a convenient shorthand for expressing 
binary values in many situations.  

For example, consider the following value expressed in binary, 
hexadecimal and decimal (spaced to show the relationship
between binary and hex):

\begin{verbatim}
Binary value:       0010 0111 1011 1010 1100 1100 1111 0101
Hex Value:             2    7    B    A    C    C    F    5
Decimal Value:                                    666553589
\end{verbatim}

Empirically we can see that grouping the bits into sets of four
allows an easy conversion to hex and expressing it as such is
$\frac{1}{4}$ as long as in binary while at the same time
allowing for easy conversion back to binary.

The decimal value in this example does not easily convey a sense
of the binary value.

%%%%%%%%%%%%%%%%%%%%%%%%%%%%%%%%%%%%%%%%%%%%%%%%%%%%%%%%%%%%%%%%%%%%%%%%%
%%%%%%%%%%%%%%%%%%%%%%%%%%%%%%%%%%%%%%%%%%%%%%%%%%%%%%%%%%%%%%%%%%%%%%%%%%
\subsection{Converting Between Bases}

%%%%%%%%%%%%%%%%%%%%%%%%%%%%%%%%%%%%%%%%%%%%%%%%%%%%%%%%%%%%%%%%%%%%%%%%%%
\subsubsection{From Binary to Decimal}
\label{section:bindec}

Alas, it is occasionally necessary to convert between decimal,
binary and/or hex.

To convert from binary to decimal, put the decimal value of the place values 
{\ldots8 4 2 1} over the binary digits like this:

\begin{verbatim}
Base-2 place values: 128 64 32 16  8  4  2  1
Binary:                0  0  0  1  1  0  1  1
Decimal:                       16 +8    +2 +1 = 27
\end{verbatim}

Now sum the place-values that are expressed in decimal for each 
bit with the value of 1: $16+8+2+1$.  The integer binary value
$00011011_2$ represents the decimal value $27_{10}$.

%%%%%%%%%%%%%%%%%%%%%%%%%%%%%%%%%%%%%%%%%%%%%%%%%%%%%%%%%%%%%%%%%%%%%%%%%%
\subsubsection{From Binary to Hexadecimal}
\label{section:binhex}

Conversion from binary to hex involves grouping the bits into
sets of four and then performing the same summing process as 
shown above.  If there is not a multiple of four bits then
extend the binary to the left with zeros to make it so.

Grouping the bits into sets of four and summing:

\begin{verbatim}
Base-2 place values:   8 4 2 1     8 4 2 1     8 4 2 1     8 4 2 1
Binary:                0 1 1 0     1 1 0 1     1 0 1 0     1 1 1 0
Decimal:                 4+2  =6   8+4+  1=13  8+  2  =10  8+4+2  =14
\end{verbatim}

After the summing, convert each decimal value to hex.  The decimal
values from 0--9 are the same values in hex.  Because we don't have any
more numerals to represent the values from 10-15, we use the first 6
letters (See the right-most column of \autoref{Figure:integers}.)
Fortunately there are only six hex mappings involving letters.  Thus 
it is reasonable to memorize them.

Continuing this example:

\begin{verbatim}
Decimal:               6           13          10          14
Hex:                   6           D           A           E
\end{verbatim}



%%%%%%%%%%%%%%%%%%%%%%%%%%%%%%%%%%%%%%%%%%%%%%%%%%%%%%%%%%%%%%%%%%%%%%%%%%
\subsubsection{From Hexadecimal to Binary}

Again, the four-bit mapping between binary and hex makes this
task as straight forward as using a look-up table.

For each \gls{hit} (Hex digIT), translate it to its unique four-bit pattern.
Perform this task either by memorizing each of the 16 patterns 
or by converting each hit to decimal first and then converting
each four-bit binary value to decimal using the place-value summing 
method discussed in \autoref{section:bindec}.

For example:

\begin{verbatim}
Hex:                     7          C
Decimal Sum:       4+2+1=7  8+4    =12
Binary:          0 1 1 1    1 1 0 0
\end{verbatim}


%%%%%%%%%%%%%%%%%%%%%%%%%%%%%%%%%%%%%%%%%%%%%%%%%%%%%%%%%%%%%%%%%%%%%%%%%%
\subsubsection{From Decimal to Binary}

To convert arbitrary decimal numbers to binary, extend the list 
of binary place values until it exceeds the value of the decimal 
number being converted.  Then make successive subtractions of each 
of the place values that would yield a non-negative result.

For example, to convert $1234_{10}$ to binary:


\begin{verbatim}
Base-2 place values: 2048-1024-512-256-128-64-32-16-8-4-2-1

   0          2048      (too big)
   1   1234 - 1024 = 210
   0          512       (too big)
   0          256       (too big)
   1    210 - 128  = 82
   1     82 - 64   = 18
   0          32        (too big)
   1     18 - 16   = 2
   0          8         (too big)
   0          4         (too big)
   1      2 - 2    = 0
   0          1         (too big)
\end{verbatim}

The answer using this notation is listed vertically
in the left column with the \acrshort{msb} on the top and 
the \acrshort{lsb} on the bottom line: $010011010010_2$.


%%%%%%%%%%%%%%%%%%%%%%%%%%%%%%%%%%%%%%%%%%%%%%%%%%%%%%%%%%%%%%%%%%%%%%%%%%
\subsubsection{From Decimal to Hex}

Conversion from decimal to hex can be done by using the place
values for base-16 and the same math as from decimal to binary
or by first converting the decimal value to binary and then
from binary to hex by using the methods discussed above.

Because binary and hex are so closely related, performing
a conversion by way of binary is quite straight forward.


%%%%%%%%%%%%%%%%%%%%%%%%%%%%%%%%%%%%%%%%%%%%%%%%%%%%%%%%%%%%%%%%%%%%%%%%%
%%%%%%%%%%%%%%%%%%%%%%%%%%%%%%%%%%%%%%%%%%%%%%%%%%%%%%%%%%%%%%%%%%%%%%%%%%
\subsection{Addition of Binary Numbers}

The addition of binary numbers can be performed long-hand the
same way decimal addition is taught in grade school.  In fact binary
addition is easier since it only involves adding 0 or 1.

The first thing to note that in any number base $0+0=0$, $0+1=1$, and 
$1+0=1$.  Since there is no ``two'' in binary (just like there is 
no ``ten'' decimal) adding $1+1$ results in a zero with a carry as
in: $1+1=10_2$ and in: $1+1+1=11_2$.  Using these five sums, any two
binary integers can be added.

For example:

\begin{verbatim}
        111111  1111  <== carries
     0110101111001111 <== addend
   + 0000011101100011 <== addend
   ------------------
     0111001100110010 <== sum
\end{verbatim}

%%%%%%%%%%%%%%%%%%%%%%%%%%%%%%%%%%%%%%%%%%%%%%%%%%%%%%%%%%%%%%%%%%%%%%%%%%
%%%%%%%%%%%%%%%%%%%%%%%%%%%%%%%%%%%%%%%%%%%%%%%%%%%%%%%%%%%%%%%%%%%%%%%%%
\subsection{Signed Numbers}

There are multiple methods used to represent signed binary integers.
The method used by most modern computers is called ``two's complement.''

A two's complement number is encoded in such a manner as to simplify
the hardware used to add, subtract and compare integers.  

A simple method of thinking about two's complement numbers is to
negate the place value of the \acrshort{msb}.  For example, the 
number one is represented the same as discussed before:

\begin{verbatim}
Base-2 place values:  -128 64 32 16  8  4  2  1
Binary:                  0  0  0  0  0  0  0  1
\end{verbatim}

The \acrshort{msb} of any negative number in this format will always
be 1.  For example the value $-1_{10}$ is:

\begin{verbatim}
Base-2 place values:  -128 64 32 16  8  4  2  1
Binary:                  1  1  1  1  1  1  1  1
\end{verbatim}

\ldots because: $-128+64+32+16+8+4+2+1=-1$.


Calculating $4+5 = 9$

\begin{verbatim}
	   1    <== carries
	 000100 <== 4
	+000101 <== 5
     ------
	 001001 <== 9
\end{verbatim}

Calculating $-4+ -5 = -9$

\begin{verbatim}
	1 11     <== carries
	  111100 <== -4
	 +111011 <== -5
   ---------
	1 110111 <== -9 (with a truncation)

  -32 16 8 4 2 1
    1  1 0 1 1 1
 -32 + 16 + 4 + 2 + 1 = -9
\end{verbatim}


This format has the virtue of allowing the same addition logic 
discussed above to be used to calculate $-1+1=0$.

\begin{verbatim}
   -128 64 32 16  8  4  2  1 <== place value
   1  1  1  1  1  1  1  1  0 <== carries
      1  1  1  1  1  1  1  1 <== addend (-1)
    + 0  0  0  0  0  0  0  1 <== addend (1)
      ----------------------
   1  0  0  0  0  0  0  0  0 <== sum (0 with a truncation)
\end{verbatim}

In order for this to work, the carry out of the sum of the MSBs is 
ignored.

%%%%%%%%%%%%%%%%%%%%%%%%%%%%%%%%%%%%%%%%%%%%%%%%%%%%%%%%%%%%%%%%%%%%%%%%%%
\subsubsection{Converting between Positive and Negative}

Changing the sign on two's complement numbers can be described as
inverting all of the bits (which is also known as the one's complement)
and then add one.

For example, inverting the number {\em four}:

\begin{verbatim}
   -128 64 32 16  8  4  2  1
      0  0  0  0  0  1  0  0 <== 4

                     1  1    <== carries
      1  1  1  1  1  0  1  1 <== one's complement of 4
    + 0  0  0  0  0  0  0  1 <== plus 1
      ----------------------
      1  1  1  1  1  1  0  0 <== -4
\end{verbatim}

This can be verified by adding 5 to the result and observe that
the sum is 1:

\begin{verbatim}
   -128 64 32 16  8  4  2  1
      1  1  1  1  1          <== carries
      1  1  1  1  1  1  0  0 <== -4
    + 0  0  0  0  0  1  0  1 <== 5
      ----------------------
   1  0  0  0  0  0  0  0  1
\end{verbatim}

Note that the changing of the sign using this method is symmetric
in that it is identical when converting from negative to positive
and when converting from positive to negative: flip the bits and
add 1.

For example, changing the value -4 to 4 to illustrate the
reverse of the conversion above:

\begin{verbatim}
   -128 64 32 16  8  4  2  1
      1  1  1  1  1  1  0  0 <== -4

                     1  1    <== carries
      0  0  0  0  0  0  1  1 <== one's complement of -4
    + 0  0  0  0  0  0  0  1 <== plus 1
      ----------------------
      0  0  0  0  0  1  0  0 <== 4
\end{verbatim}

%%%%%%%%%%%%%%%%%%%%%%%%%%%%%%%%%%%%%%%%%%%%%%%%%%%%%%%%%%%%%%%%%%%%%%%%%
%%%%%%%%%%%%%%%%%%%%%%%%%%%%%%%%%%%%%%%%%%%%%%%%%%%%%%%%%%%%%%%%%%%%%%%%%%
\subsection{Subtraction of Binary Numbers}


Subtraction of binary numbers is performed by first negating
the subtrahend and then adding the two numbers.  Due to the
nature of two's complement numbers this will work for both 
signed and unsigned numbers.
\enote{This section needs more examples of subtracting 
signed an unsigned numbers and a discussion on how 
signedness is not relevant until the results are interpreted. 
For example adding $-4+ -8=-12$ using two 8-bit numbers 
is the same as adding $252+248=500$ and truncating the result 
to 244.}

To calculate $-4-8 = -12$

\begin{verbatim}
   -128 64 32 16  8  4  2  1
      1  1  1  1  1  1  0  0 <== -4
    - 0  0  0  0  1  0  0  0 <== 8


                  1  1  1    <== carries
      1  1  1  1  0  1  1  1 <== one's complement of -8
    + 0  0  0  0  0  0  0  1 <== plus 1
      ----------------------
      1  1  1  1  1  0  0  0 <== -8
      
	  
      1  1  1  1             <== carries
      1  1  1  1  1  1  0  0 <== -4
    + 1  1  1  1  1  0  0  0 <== -8
      ----------------------
   1  1  1  1  1  0  1  0  0 < == -12
\end{verbatim}


%%%%%%%%%%%%%%%%%%%%%%%%%%%%%%%%%%%%%%%%%%%%%%%%%%%%%%%%%%%%%%%%%%%%%%%%%
%%%%%%%%%%%%%%%%%%%%%%%%%%%%%%%%%%%%%%%%%%%%%%%%%%%%%%%%%%%%%%%%%%%%%%%%%
\subsection{Truncation and Overflow}

Discuss the details of truncation and overflow here.
\enote{This chapter should be made consistent in its use of 
{\em truncation} and {\em overflow} as occur with signed and unsigned
addition and subtraction.}

I prefer to define {\em truncation} as the loss of data as result of 
the bit-length of the destination being too small to hold result of an 
operation and {\em overflow} as when the carry into a sign bit is not 
the same as the carry out of the sign bit.

Where addition and subtraction on the RV32 is concerned, the sum or difference of 
two unsigned 32-bit numbers will be {\em truncated} when the operation results in 
a carry out of bit 31.  Unsinged operations can not overflow (as defined above).

(show a truncation picture here)

An Overflow occurs with signed numbers when the two addends are positive and 
sum is negative or the addends are both negative and the sum is positive.

(show an overflow picture here)

(show mixed overflow and truncation situations here to drive home the need
to ignore truncation when dealing with signed numbers.)

0xffffffff + 0x00000002 has truncation but not overflow 
(OK for signed, not OK for unsigned).

0xffffffff + 0xfffffffe also has truncation but not overflow.

0x40000000 + 0x40000000 has overflow but not truncation. (We care if are signed numbers.)

0x80000000 + 0x80000000 has both overflow and truncation. (we care regardless of signedness)

Where subtraction is concerned the notion of a borrow is the same as carry.


\enote{I think that overloading the word {\em overflow} like this can be is 
confusing to new programmers.}%
Page 13 of~\cite{rvismv1v22:2017} mixes these two notions of 
(and never mentions the word {\em truncate}) like this:
\begin{quote}
We did not include special instruction set support for overflow checks on 
integer arithmetic operations in the base instruction set, as many overflow 
checks can be cheaply implemented using RISC-V branches. Overflow checking for 
unsigned addition requires only a single additional branch instruction after the 
addition: \verb@add t0, t1, t2; bltu t0, t1, overflow@.  

For signed addition, if one operand's sign is known, overflow checking requires 
only a single branch after the addition: 
\verb@addi t0, t1, +imm; blt t0, t1, overflow@. This covers the common 
case of addition with an immediate operand.

For general signed addition, three additional instructions after the addition 
are required, leveraging the observation that the sum should be less than one 
of the operands if and only if the other operand is negative.
\begin{verbatim}
add t0, t1, t2
slti t3, t2, 0
slt t4, t0, t1
bne t3, t4, overflow
\end{verbatim}
In RV64, checks of 32-bit signed additions can be optimized further by comparing 
the results of ADD and ADDW on the operands.
\end{quote}



%%%%%%%%%%%%%%%%%%%%%%%%%%%%%%%%%%%%%%%%%%%%%%%%%%%%%%%%%%%%%%%%%%%%%%%%%
%%%%%%%%%%%%%%%%%%%%%%%%%%%%%%%%%%%%%%%%%%%%%%%%%%%%%%%%%%%%%%%%%%%%%%%%%
\section{Sign and Zero Extension}

\index{sign extension}
\label{SignExtension}
Due to the nature of the two's complement encoding scheme, the following 
numbers all represent the same value:
\begin{verbatim}
                                   1111 <== -1
                               11111111 <== -1
                   11111111111111111111 <== -1
           1111111111111111111111111111 <== -1
\end{verbatim}
As do these:
\begin{verbatim}
                                  01100 <== 12
                             0000001100 <== 12
    00000000000000000000000000000001100 <== 12
\end{verbatim}

The phenomenon illustrated here is called {\em sign extension}.  That is
any signed number can have any quantity of additional MSBs added to it,
provided that they repeat the value of the sign bit.

\autoref{Figure:SignExtendNegative} illustrates extending the negative sign
bit of {\em val} to the left by replicating it.
When {\em val} is negative, its \acrshort{msb} (bit 19 in this example) will
be set to 1.   Extending this value to the left will set all the new bits
to the left of it to 1 as well.

\begin{figure}[ht]
\centering
\DrawBitBoxSignExtendedPicture{32}{10100000000000000010}
\captionof{figure}{Sign-extending a negative integer from 20 bits to 32 bits.}
\label{Figure:SignExtendNegative}
\end{figure}

\autoref{Figure:SignExtendPositive} illustrates extending the positive sign
bit of {\em val} to the left by replicating it.
When {\em val} is positive, its \acrshort{msb} will be set to 0.  Extending this
value to the left will set all the new bits to the left of it to 0 as well.

\begin{figure}[ht]
\centering
\DrawBitBoxSignExtendedPicture{32}{01000000000000000010}
\captionof{figure}{Sign-extending a positive integer from 20 bits to 32 bits.}
\label{Figure:SignExtendPositive}
\end{figure}


\label{ZeroExtension}
In a similar vein, any {\em unsigned} number also may have any quantity of 
additional MSBs added to it provided that htey are all zero.  For example,
the following all represent the same value:
\begin{verbatim}
                                   1111 <== 15
                                  01111 <== 15
          00000000000000000000000001111 <== 15
\end{verbatim}

The observation here is that any {\em unsigned} number may be
{\em zero extended} to any size.

\autoref{Figure:ZeroExtend} illustrates zero-extending a 20-bit {\em val} to the
left to form a 32-bit fullword.

\begin{figure}[ht]
\centering
\DrawBitBoxZeroExtendedPicture{32}{10000000000000000010}
\captionof{figure}{Zero-extending an unsigned integer from 20 bits to 32 bits.}
\label{Figure:ZeroExtend}
\end{figure}

%Sign- and zero-extending binary numbers are common operations used to 
%fit a byte or halfword into a fullword.

%%%%%%%%%%%%%%%%%%%%%%%%%%%%%%%%%%%%%%%%%%%%%%%%%%%%%%%%%%%%%%%%%%%%%%%%%
%%%%%%%%%%%%%%%%%%%%%%%%%%%%%%%%%%%%%%%%%%%%%%%%%%%%%%%%%%%%%%%%%%%%%%%%%
\section{Shifting}

We were all taught how to multiply and divide decimal numbers by ten
by moving (or {\em shifting}) the decimal point to the right or left 
respectively.  Doing the same in any other base has the same effect 
in that it will multiply or divide the number by the value of the base.

Multiplication and division are only two reasons for shifting.  There
can be other occasions where doing so is useful.

As implemented by a CPU, shifting applies to the value in a register
and the results stored back into a register of finite size.  Therefore
a shift result will always be truncated to fit into a register.

Note that when dealing with numeric values, any truncation performed 
during a right-shift will manifest itself as rounding toward zero.

\subsection{Logical Shifting}

Shifting {\em logically} to the left or right is a matter of re-aligning
the bits in a register and trncating the result.

\enote{Redraw these with arrows tracking the shifted bits and the truncated values}%
To shift left two positions:

\DrawBitBoxUnsignedPicture{10111000000000000010}\\
\DrawBitBoxUnsignedPicture{11100000000000001000}

To shift right one position:

\DrawBitBoxUnsignedPicture{10111000000000000010}\\
\DrawBitBoxUnsignedPicture{01011100000000000001}

Note that the vacated bit positions are always filled with zero.

\subsection{Arithmetic Shifting}

Some times it is desirable to retain the value of the sign bit when
shifting.  The RISC-V ISA provides an arithmetic right shift 
instruction for this purpose (there is no arithmetic left shift for 
this ISA.)

When shifting to the right {\em arithmetically}, vacated bit positions are
filled by replicating the value of the sign bit. 

An arithmetic right shift of a negative number by 4 bit positions:

\DrawBitBoxSignedPicture{10111000000000000010}\\
\DrawBitBoxSignedPicture{11111011100000000000}

%%%%%%%%%%%%%%%%%%%%%%%%%%%%%%%%%%%%%%%%%%%%%%%%%%%%%%%%%%%%%%%%%%%%%%%%%
%%%%%%%%%%%%%%%%%%%%%%%%%%%%%%%%%%%%%%%%%%%%%%%%%%%%%%%%%%%%%%%%%%%%%%%%%
\section{Main Memory Storage}

As mentioned in \autoref{VolatileStorage}, the main memory in a RISC-V 
system is byte-addressable.  For that reason we will visualize it by 
displaying ranges of bytes displayed in hex and in \gls{ascii}.  As will 
become obvious, the ASCII part makes it easier to find text messages.%
\footnote{Most of the memory dumps in this text are generated by \gls{rvddt}
and are shown on a per-byte basis without any attempt to reorder their
values. Some other applications used to dump memory do not dump the bytes
in address-order!  It is important to know how your software tools operate
when using them to dump the contents of memory and/or files.}

%%%%%%%%%%%%%%%%%%%%%%%%%%%%%%%%%%%%%%%%%%%%%%%%%%%%%%%%%%%%%%%%%%%%%%%%%
\subsection{Memory Dump}

\listingRef{rvddt_memdump.out} shows a memory dump from the rvddt
`d' command requesting a dump starting at address \hex{00002600}
for the default quantity (\hex{100}) of bytes.

\listing{rvddt_memdump.out}{{\tt rvddt} memory dump}

\begin{itemize}
\item [$\ell$ 1] The rvddt prompt shwing the dump command.
\item [$\ell$ 2] From left to right. the dump is presented as the address 
	of the first byte (\hex{00002600}) followed by a colon, the value
	of the byte at address \hex{00002600} expressed in hex, the next byte
	(at address \hex{00002601}) and so on for 16 bytes. There is a dash
	between the 7th and 8th bytes to help provide a visual reference for
	the center to make it easy to locate bytes on the right end.  For 
	example, the byte at address \hex{0000260c} is four bytes to the 
	right of byte number eight (at the dash) and contains \hex{13}.
	To the right of the 16-bytes is an asterisk-enclosed set of 16 columns
	showing the ASCII characters that each byte represents.  If a byte
	has a value that corresponds to a printable character code, the character
	will be displayed.  For any illegal/un-displayable byte values, a dot 
	is shown to make it easier to count the columns.
\item [$\ell$ 3-17] More of the same as seen on $\ell$ 2.  The address
	at the left can be seen to advance by $16_{10}$ (or $10_{16}$) 
	for each line shown.
\end{itemize}

%%%%%%%%%%%%%%%%%%%%%%%%%%%%%%%%%%%%%%%%%%%%%%%%%%%%%%%%%%%%%%%%%%%%%%%%%
\subsection{Endianness}

The choice of which end of a multi-byte value is to be stored at the
lowest byte address is referred to as {\em endianness.}  For example,
if a CPU were to store a \gls{halfword} into memory, should the byte 
containing the \acrfull{msb} (the {\em big} end) go first or does 
the byte with the \acrfull{lsb} (the {\em little} end) go first/into 
the lowest memory address?

On the one hand the choice is arbitrairy.  On the other hand, it is 
possible that the choice could impact the performance of the system.%
\footnote{See\cite{IEN137} for some history of the big/little-endian ``controversy.''}

IBM mainframe CPUs and the 68000 family store their bytes in big-endian 
order.  While the Intel Pentium and most embedded processors are little 
endian.  Some CPUs are even {\em bi-endian} in that they instructions that
can change their order on the fly. 

The RISC-V system uses the little-endian byte order.

%%%%%%%%%%%%%%%%%%%%%%%%%%%%%%%%%%%%%%%%%%%%%%%%%%%%%%%%%%%%%%%%%%%%%%%%%
\subsubsection{Big-Endian}
\label{BigEndian}

Using the contents of \listingRef{rvddt_memdump.out}, a {\em big-endian}
CPU would recognize the contents as follows:

\begin{itemize}
\item The 8-bit value stored at address \hex{00002658} is \hex{76}.
\item The 16-bit value stored at address \hex{00002658} is \hex{7661}.
\item The 32-bit value stored at address \hex{00002658} is \hex{76616c3d}.
\end{itemize}

Observe that the bytes in the dump are in the same order as they would
be used by the CPU if it were to read them as a multi-byte value.

%%%%%%%%%%%%%%%%%%%%%%%%%%%%%%%%%%%%%%%%%%%%%%%%%%%%%%%%%%%%%%%%%%%%%%%%%
\subsubsection{Little-Endian}
\label{LittleEndian}

Using the contents of \listingRef{rvddt_memdump.out}, a {\em little-endian} 
CPU would recognize the contents as follows:

\begin{itemize}
\item The 8-bit value stored at address \hex{00002658} is \hex{76}.
\item The 16-bit value stored at address \hex{00002658} is \hex{6176}.
\item The 32-bit value stored at address \hex{00002658} is \hex{3d6c6176}.
\end{itemize}

Observe that the bytes in the dump are in backwards order as they would
be used by the CPU if it were to read them as a multi-byte value.

Note that in a little-endian system, the number of bytes used to represent
the value does not change the place value of the first byte(s).  In this
example, the \hex{76} at address \hex{00002658} is the least significant
byte in all representations.  

In the Risc-V ISA it is noted that ``A minor point is that we have also found 
little-endian memory systems to be more natural for hardware 
designers. However, certain application areas, such as IP networking, operate
on big-endian data structures, and so we leave open the possibility of 
non-standard big-endian or bi-endian systems.''\cite[p.~6]{rvismv1v22:2017}

%%%%%%%%%%%%%%%%%%%%%%%%%%%%%%%%%%%%%%%%%%%%%%%%%%%%%%%%%%%%%%%%%%%%%%%%%
\subsection{Arrays and Character Strings}

While Endianness defines to how single values are stored in memory,
the {\em array} defines how multiple values are stored.

An array is a data structure comprised of an ordered set of elements.
This text will limit its definition of {\em array} to those sets
of elements that are all of the same {\em type}.  Where {\em type}
refers to the size (number of bytes) and representation (signed,
unsigned) of each element.

In an array, the elements are stored adjacent to one another such that the
address of any element may be defined as:

\begin{equation}
e = a + n * s
\end{equation}

Where $n$ is the element number of interest, $e$ is the address of element 
of interest, $a$ is the address of the first element in the array, $s$ 
is the size of each element, $a[0]$ is the first element of the array
and $a[n-1]$ is the last element of the array.%
\footnote{Some computing languages (C, C++, Java, C\#, Python, Perl,\ldots) 
define an array such that the first element is indexed as $a[0]$.  
While others (FORTRAN, MATLAB) define the first element of an 
array to be $a[1]$.}

Using this definition, \listingRef{rvddt_memdump.out}, knowledge that 
we are using a little-endian machine and given that
$a = $\hex{00002656} and $s = 2$, the values of the first 8 elements 
of array $a$ are:

\begin{itemize}
\item $a[0]$ is \hex{0000} and is stored at \hex{00002656}.
\item $a[1]$ is \hex{6176} and is stored at \hex{00002658}.
\item $a[2]$ is \hex{3d6c} and is stored at \hex{0000265a}.
\item $a[3]$ is \hex{0000} and is stored at \hex{0000265c}.
\item $a[4]$ is \hex{0000} and is stored at \hex{00002660}.
\item $a[5]$ is \hex{0000} and is stored at \hex{00002662}.
\item $a[6]$ is \hex{8480} and is stored at \hex{00002664}.
\item $a[7]$ is \hex{412e} and is stored at \hex{00002666}.
\end{itemize}

As a general rule, there is no fixed rule or notion as to how many 
elements an array has.  It is up to the programmer to ensure that
the starting address and the nubmer of elements in any given array
(its size) are used properly so that data bytes outside an array
are not accidently used as elements.

There is, however, a common convention used for an array of 
characters that is used to hold a text message 
(called a {\em character string} or just {\em string}).

When an array is used to hold a string the element past the last
character in the string is set to zero.  This is because 1) zero 
is not a valid printable ASCII character and 2) it simplifies
software in that knowing no more than the starting address of a
string is all that is needed to processes it.  Without this zero
{\em sentinel} value (called a {\em null} terminator), some knowledge
of the number of characters in the string would have to otherwise 
be conveyed to any code needing to consume or process the string.

In \listingRef{rvddt_memdump.out}, the 5-byte long array starting 
at address \hex{00002658} contains a string whose value can be
expressed as either of:

\begin{itemize}
\item \verb@76 61 6c 3d 00@
\item \verb@"val="@
\end{itemize}

\index{ASCII}
\index{ASCIIZ}
When the double-quoted text form is used, the GNU assembler used in 
this text differentiates between {\em ascii} and {\em asciiz} strings
such that an ascii string is {\em not} null terminated and an
asciiz string {\em is} null terminated.  

The value of providing a method to create a string that is {\em not} 
null terminated is that a program may define a large string by 
concatenating a number of ascii strings together and following the 
last with a byte of zero to null-terminate the lot.

It is a common mistake to create a string with a missing
null terminator.  The result of printing such a ``string'' is that
the string is printed and as well as whatever random data bytes in 
memory that follows it until a byte whose value is zero is found 
by chance.

%%%%%%%%%%%%%%%%%%%%%%%%%%%%%%%%%%%%%%%%%%%%%%%%%%%%%%%%%%%%%%%%%%%%%%%%%
\subsection{Context is Important!}

Data values can be interpreted differently depending on the context in 
which they are used.  Assuming what a set of bytes is used for based on 
their contents can be very misleading!  For example, there is a 0x76 at 
address 0x00002658.  This is a `v' is you use it as an ASCII 
(see~\autoref{chapter:ascii}) character, a $118_{10}$ if it is an integer 
value and TRUE if it is a conditional.

%%%%%%%%%%%%%%%%%%%%%%%%%%%%%%%%%%%%%%%%%%%%%%%%%%%%%%%%%%%%%%%%%%%%%%%%%
\subsection{Alignment}

\enote{Include the obligatory diagram showing the overlapping data types 
when they are all aligned.}%
With respect to memory and storage, {\em \gls{alignment}} refers to the 
{\em location} of a data element when the address that it is stored is 
a precise multiple of a power-of-2. 

The primary alignments of concern are typically 2 (a halfword), 
4 (a fullword), 8 (a double word) and 16 (a quad-word) bytes.

For example, any data element that is aligned to 2-byte boundary
must have an (hex) address that ends in any of: 0, 2, 4, 6, 8, A, 
C or E.
Any 4-byte aligned element must be located at an address ending
in 0, 4, 8 or C.  An 8-byte aligned element at an address ending
with 0 or 8, and 16-byte aligned elements must be located at
addresses ending in zero.

Such alignments are important when exchanging data between the CPU 
and memory because the hardware implementations are optimized to
transfer aligned data.  Therefore, aligning data used by any program
will reap the benefit of running faster.

An element of data is considered to be {\em aligned to its natural size}
when its address is an exact multiple of the number of bytes used to 
represent the data.  Note that the ISA we are concerned with {\em only}
operates on elements that have sizes that are powers of two.

For example, a 32-bit integer consumes one full word.  If the four bytes
are stored in main memory at an address than is a multiple of 4 then
the integer is considered to naturally aligned.  

The same would apply to 16-bit, 64-bit, 128-bit and other such values 
as they fit into 2, 8 and 16 byte elements respectively.

Some CPUs can deliver four (or more) bytes at the same time while others 
might only be capable of delivering one or two bytes at a time.  Such 
differences in hardware typically impact the cost and performance of a 
system.%
\footnote{The design and implementation
choices that determine how any given system operates are part of what is
called a system's {\em organization} and is beyond the scope of this text.
See~\cite{codriscv:2017} for more information on computer organization.}


%%%%%%%%%%%%%%%%%%%%%%%%%%%%%%%%%%%%%%%%%%%%%%%%%%%%%%%%%%%%%%%%%%%%%%%%%
\subsection{Instruction Alignment}

The RISC-V ISA requires that all instructions be aligned to their
natural boundaries.

Every possible instruction that an RV32I CPU can execute contains
exactly 32 bits.  Therefore they are always stored on a full word
boundary.  Any {\em unaligned} instruction is {\em illegal}.%
\footnote{This rule is relaxed by the C extension to allow an 
instruction to start at any even address.\cite[p.~5]{rvismv1v22:2017}}

An attempt to fetch an instruction from an unaligned address
will result in an error referred to as an alignment {\em \gls{exception}}.
This and other exceptions cause the CPU to stop executing the
current instruction and start executing a different set of instructions
that are prepared to handle the problem.  Often an exception is
handled by completely stopping the program in a way that is commonly
referred to as a system or application {\em crash}.
